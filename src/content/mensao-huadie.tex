\chapter[化蝶]{化\qquad 蝶}
\blfootnote{大学毕业之后,大约一九九五年前后。}

庄子曾经做了一个梦,梦见了自己变成了蝴蝶。

蝴蝶曾经做了一个梦,梦见了自己变成了庄子。

上面的两句话,第一句好懂,第二句着实有些让人莫不着头脑。当初,庄子或者蝴蝶或者庄子和蝴蝶也不知道第一句是真还是第二句是真。

江南吴越地方的传说却是有化蝶的,不是如此地不明白,而是明明白白地化蝶。这便是著名的民间传说:梁山伯与祝英台。不过,我很奇怪的是两位主人公会有这样奇怪的名字。年纪轻轻何以敢称“伯”,若说是排行也应该在中间。“英台”二字,知识隐约觉着从南朝某处诗歌中得来,手头没有资料,想考证也只好作罢。传说中祝英台之父是否是饱学大儒,也没有明言,不过说得明明白白的其父是个员外。“英台”二字好象也就没有来由了。也许以讹化讹,便这样说开了;话说回来,吴越的方言也着实难懂,就同音替代也未可知。我想也不必象非要找出“杞良妻”就是“孟姜女”的来源那样。大家都这样说,便这样说罢。

梁山伯与祝英台的传说,在吴越地方不只是大家所熟知的,还有许许多多的故事,要是都说出来大约可以写成大部的砖头书。只是其中之一我颇感兴趣。说是祝英台有一个哥哥,有哥哥就少不了有嫂子。大多数的民间传说中,姑嫂照例是合不来的,这里也没有例外。既然祝英台是个“美丽动人的好姑娘”,嫂子当然是个“恶毒凶狠的坏女人”。这里我要声明:这是照着老例讲的;至于,祝英台是否真是个“美丽动人的好姑娘”,嫂子是否真是个“恶毒凶狠的坏女人”,这里暂且这么说罢。

祝英台要去杭州读书。她父亲是反对的,嫂子也是反对的。不过,既然其父爱其女如掌上明珠,也是扭不过的。嫂子反对的理由是:一个未出阁的女子,如此便是不守妇道要如何如何了。为了不象方鸿渐那样如何了女学生的耳朵,这里就不明说了,恐怕有人比我还要明白得多得多。祝英台便指着后园的一丛牡丹说:花败便如何等等。嫂子自然是不服气的,祝英台一走,嫂子就用开水烫牡丹,不但没有烫坏,反而越发地茂盛了。情急之下拔出根来用火烧,再插回地里去,地上的部分是不能动的,否则是要露馅的。没想到烧过之后不但没死,居然烧出个“焦骨牡丹”的新品种来。祝英台的嫂子做了一次牡丹品种改良的大功劳。要由此来说明祝英台的冰清而玉洁。不过,这里有个可疑之处,牡丹从不生于江南,祝英台的籍贯便不会是吴越地方了。

前面说了祝英台要去杭州读书。要读书非去杭州不可,俗话说得好:“上有天堂,下有苏杭。”杭州是繁华闹市,旅游胜地。当然现在的浙江大学是很好的学校,但当时呢?不得而知。不过,大儒先哲结庐于偏僻之处,可以推而及之,杭州准确地说当时的杭州是否真是个读书的好去处。那么祝英台要去杭州读书的目的,大家自己去想象罢。

回过头来看看梁山伯。至于梁山伯,我不知道,他是真傻还是假傻。若是真傻,便是迂腐,这样的人本不值得爱的;若是假傻,便是情场之高手了,按现代的说法叫做“吊”。十里相送,祝英台以九妹为掩暗许终生,梁山伯欣喜若狂;而祝英台所指的九妹便是自己,梁山伯若是以为祝英台为男子, 那么凭什么理由认为“九妹”就是个好姑娘,能够成为自己的好妻子。梁山伯若是假傻,其手段是高明的,一方面既符合了道德的规范,一方面又有了佳人为妻,两头两头逢源;这样的话,十里相送,缠绵悱恻,倒是非常有趣的。

祝英台的父亲,传说中是个员外,员外其实就是地主老财,决不会是个饱学大儒。爱其女如掌上明珠,好象也是没有理由的;若是真的爱其女就不会因为县令的压力,而应承这门婚事。看来,爱财才是真的,爱其女不过只是当其是一个玩偶罢了。祝英台作为一个女子违反礼教,这件事,是当时的道德规范所不允许的;因而,祝英台的嫂子的反对当然是有理由的,而且是正当的理由。其实,牡丹根本不用去烧,“嫁出门的姑娘,泼出门的水。”祝英台不会分其家产,反正妆奁是少不了的,清白也好不清白也好于她是乌有关系的。只是这传说要说祝英台的好,定要找个局外人来垫底罢。

令我奇怪的另一件是:既然有了楼台会,想必是可以私奔的。司马相如和卓文君的私奔是个佳话,那么,梁山伯和祝英台的私奔也应该是个佳话。结果呢?梁山伯郁郁而亡,祝英台投进了坟墓。不过,如果真的是马家迎亲,壮丁一定少不了,正要趁着机会巴结县太爷,如何连一个弱女子也拦不住,祝英台毕竟没有祥林嫂那样的健壮。于是,祝英台投进了坟墓,与梁山伯双双化为了蝴蝶,翩翩起舞了。

写到这儿,我想起了与此相类似的一个西方著名悲剧《罗密欧与朱丽叶》。结果一样都是一个“死”字,但导致“死”的原因却有天壤之别。朱丽叶的假死是要躲过出嫁,其目的是要活下去,和罗密欧偕老的。只是阴差阳错导致两人双双殉情。要注意的是:朱丽叶把“死”作为一种手段,以达到和罗密欧的结合;而梁山伯看来根本没有打算活下去,这是一个多么大的差距。生命是可贵的,如果没有了生命,还能够谈论什么?更不要说“爱情”了。当然牺牲是不可避免,也不应该避免,那是为了更多的生命。就个人而言,生命难道不是最可宝贵的吗?

然而,人的意志如何坚强,也是难以承受岁月的磨砺。因此,没有人能够在没有结果的爱情之后,说“永爱一人”。故事在此以“死”作为终结,是再好不过的了,坚贞不渝的爱情便越发显得高尚了。人的生命相较于此,也就无可宝贵的了。“死”便成为一种逃避,逃避美丽爱情故事后面难以掩饰的尴尬境地。

国人的自我安慰,的确是举世无双的。现实里不能够实现的,可以等到来世。生不能为夫妻,死后为蝴蝶,世人便管不着了。逃避现实,国人的经历太多了。

当然,梁祝的故事主题是一个几乎永恒的主题——爱情。宝黛的爱情,得出了一个“空”字,我真的不明白,象贾宝玉这样一个热爱生活的人,居然会跟随和尚道士离家出走。梁祝的爱情得到的是一个虚幻的团圆。人类的想象是任何势力所不能遏止的,于是,想象便是一个绝好的去处。现实是残酷的,想象可以按照人们的意志去安排。我不知道,玉皇大帝是否会干涉这对可爱可怜可敬可悲而又太脆弱的一对蝴蝶的爱情。

梁祝追求的是否真是人们所说的“自由恋爱”呢?至少在我看来不是这样的。祝英台并没有直接说出来,而是假托并不存在的九妹,归根到底是要符合礼教的,不恰当地套用一句话叫做“曲线救国”。如果成功了,便是佳话,“才貌双全,珠联璧合”;如果不成功,便是悲剧。然而,悲剧却长了一条不折不扣的老鼠尾巴——化蝶,所有的矛盾就这样解决了,倒也快刀斩乱麻;无论怎样,在人们悲愤之余,来一条手绢擦擦眼泪为化蝶而欣喜,总是件好事。于是,悲剧就这样地淡化了,传说也因此而美丽起来。当然杭州的美丽景色为这个美丽的传说又添了一层美丽的外衣,恰如一个愤怨的美人,越发地标致起来。人们便忘记了美人的愤怨,只是看着美人的容貌发呆。

倘若当初司马相如和卓文君的故事也是以悲剧而终结的话,按照国人的传统必然有个同样的类似化蝶的尾巴。

我们所需要的不是幻想而是现实和由现实推及的理想。现实需要承认,但现实更需要我们去改变。梁祝充其量不过是企图钻一个现实的空子,实际上企图没有实现。倘若,梁祝真正是为了自己的爱情,私奔应该是最好的解决办法。可是对于梁山伯与祝英台来说,这是一个灾难,祝英台能否象卓文君一样当垆卖酒,梁山伯能否去做店小二,我是不敢说的。建筑在空虚的基础上的爱情,是无法存在下去的。设想梁祝二人如果真的“有情的人都成了眷属”,浪漫的爱情只可能建立在衣食不愁的基础上,只可能是有钱而有闲的。

马文才这个角色,不应该不提的。传说中的马文才,相貌猥琐,胸无点墨,是个仗势欺人的额势力的代表。马文才的出现不过是要悲剧的悲伤更浓一点,让人们的泪掉得更多一点,从而使那条老鼠尾巴看起来更漂亮的一点而已;即使没有马文才这个角色,梁祝的故事就能够圆满吗?

其实,梁祝的故事并不一定有过,只是国人逃避现实的又一个顶峰。可以安慰许多并不幸福的人把幸福寄托在天国,寄托在来世;可以省却许多反抗和反抗带来的混乱。不能不说,为了太平而叫人们把追求放在谁也不能证明的幻想之中,的确是个好方法。直到现在还是有人这样去做。

社会进步了,思想解放了,天国的幻想也就没落了,被人们抛弃了;然而人们把理想也一同抛进了垃圾堆,只在注意“利”字。有钱的浪漫成了时髦,有钱的潇洒成了时尚;钱便决定了一切,决定了生命。人们打破了礼教的枷锁,又给自己套上了一个铜臭的枷锁。而国人的眼光却没有多少变化,能够打动人心的仍然是那些“不食人间烟火”的“金童玉女”之间的“爱情”,是美丽的,是缠绵的;传说中的美人更加妩媚了,更加挑逗人们心中的那一块小秘密了,挑逗得人们越发地舒心顺气。悲剧的 影子也不存在了,矛盾也不存在了,翻过来倒过去, 只剩下了“情挑”二字。

于是,少男少女纷纷追着时尚去了。悲剧就这样从我们的生活当中悄然而去。梁祝的故事只是故事而已,留不下什么东西了,人们谈论起来只是一个美丽的民间故事,一个愤怨的美人而已。其唯可利用的不外乎两种:换汤不换药和换药不换汤。梁山伯成了武林高手,英雄救美的故事,依照传统又演绎下去了,给“天下熙熙,皆为利去;天下攘攘,皆为利往”的人们增添一些笑料,让人们在笑的时候,忘记一只在自己口袋里掏钱的手,这便是成功了。于是,想象力便是捞钱的 本领。苦苦追求理想的人便成了“呆头鹅”。

梁祝的故事到化蝶便完了。我们来设想一下,这后面会发生一些什么事情。蝴蝶是否就自由了,不必说天上的玉皇大帝,来了一只黄雀,吃了一只蝴蝶,这样故事该如何演下去?是不是,另一只蝴蝶也要自投雀口?然后再化作什么呢?只能化作鸟粪了。我想不会有人允许我这样设想下去的,我自己也是不能够的。看来人们的想象还没有达到一个一劳永逸的地步,化作神仙不是更好。我不知道这样是否还能够象化蝶一样动人。

面对这样的一个故事,我们得到了什么呢?我是不知道的。悲剧没有了,只剩下了美人。逃避仍然存在,只是不能逃往化蝶的幻想之中去了。“发财”的幻想越来越深入人心了,于是人们更加“岌岌可危”地护住自己的口袋,而伸手到别人的口袋里掏钱的想象力越发地膨胀起来,落到个人身上的责任就越来越少了。

梁祝的悲哀已没有多少人理会,化蝶的传说,就成了博物馆里的精品了。在欣赏之余,可以作为“高雅”的象征。梁祝的传说,当初何曾高雅过,民间传说,村鄙俗语罢。

也许,果真是两只翩翩起舞的蝴蝶做了这场梦。庄子若是有灵,该有何想?我只好说不得而知了。
