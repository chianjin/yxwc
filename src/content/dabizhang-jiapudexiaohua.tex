\chapter{家谱的笑话}
\blfootnote{* 2006年10月7日发表于《新语丝》。}
\blfootnote{* 申明:本人对本文全部文字完全负责,与方舟子以及新语丝网站无关。}
\blfootnote{* 本文网址 http://xys.org/xys/ebooks/others/science/dajia7/jiapu.txt}

\subtitle{——评张功耀的家谱}

张先生功耀在自己的博客\footnote{张功耀原文网址 http://zhgybk.blog.hexun.com/3978286\_d.html}
上开列一份家谱:黄帝(汉族始祖)→挥(张姓始祖)→昧→台骆→伊源→候→立方→
坤→敦吾→郊→重熙→吴光→天杰→钦若→榆→临→宜→阳→安→考→承→喾→琦→希→燧→秦→还→纯→质→康→启→立→瑰→
秩→庖→颢→洙→逸→都→助→须→圆→萧→昶→浚→惠→谊→稳→元→正→炳→辰→本→灼→兖→灵→宏→道→仲→逸→伯谦→
信明→实→禹臣→元驭→熙→元叔→奉义→高陵→宜武→侯→老→君臣→趯→骼→进明→孟谈→抑朔→开地→平→
良(西汉留侯)→不疑→典→默→金→万雅→明→国真→箕→壮→凤→允→皓→宇→才→忠→某→孟成→肃→平→茂→骏→华
→韪→桂→星光→品端→轩→次恭→永能→甫之→宏简→纶→隆→应春→子犯→金→俊→守礼→君政→子胄→宏愈→
九龄(唐朝宰相)→拯→藏器→敦庆→景重→理→相→焴→登秀→廷杰→涉→宏简→载→景昌→端→仲祥→宾国→扬国→
化孙→祥云→腾辉→昭上→昊渐→敏承→先俸→君绍→启明→文宗→仲良→艺兴→山贤→崇森→隆凤→士乔→问仁→嘉元→
习孔→乾彩→可均→乔山→月老→永发→俊魁→才贤→声祥→继富→树清→功耀。

张先生据此宣称:“张功耀是西汉留侯张良第89代孙。”根据张先生的这份家谱,
我以为那些“张先生反传统”的说法是不对的。张先生显然热爱传统,大大的热
爱传统,不是每个人都能把自己的祖先象张先生一样排列清楚的。看到这份家
谱,张先生热爱传统之心跃然纸上。

张先生把祖先追溯到张九龄这个唐朝宰相,我没意见;追溯到汉代开国名臣张
良,我也没意见。毕竟都姓张。问题是张良得父祖姓张吗?《史记·留侯世家》
这样说的:“良尝学礼淮阳。东见仓海君。得力士,为铁椎重百二十斤。秦皇
帝东游,良与客狙击秦皇帝博浪沙中,误中副车。秦皇帝大怒,大索天下,求
贼甚急,为张良故也。良乃更名姓,亡匿下邳。”其中五个字“良乃更名姓”透露
一点消息,张良曾经更改过名和姓。那么“张良”这个名字是更改前的还是更改
后的?来看看《史记·留侯世家》头一段:“留侯张良者,其先韩人也。大父开
地,相韩昭侯、宣惠王、襄哀王。父平,相厘王、悼惠王。”张良的父祖都是韩
的相国。当时不像唐代以后有科举,白衣亦能取仕,当时诸侯国的卿仕必定是
贵族,相这一职务绝大多数情况下由公族担任。所谓公族就是王族。《史记索
隐》这样注释:“汉书云字子房。按:王符、皇甫谧并以良为韩之公族,姬姓也。
秦索贼急,乃改姓名。而韩先有张去疾乃张谴,恐非良之先代。”张良原来姓姬,
韩也姓姬,都是周的公族,所以张良父祖五世相韩在当时是自然的。如果张先生
的家谱就在张良处打住,我也就没有什么话可说了。历史上,改姓成为始祖的例
子很多,很正常的一件事情。

张先生并没有就此打住,还要往上,一直追溯到“黄帝”这个传说中的人物。这
一追溯怎么看怎么别扭,黄帝这个传说人物是否存在还有天大的疑问,考据来
考据去,总是没有一个子丑寅卯来,考古学上更是没有任何证据,张先生竟然
堂而皇之拿来作为自己始祖。可惜,真是替张先生可惜,按照张先生一贯的
“科学”作风,怎么也不应该啊。

就算承认黄帝是存在的,那么“挥”这个张姓始祖是黄帝之子还是之孙?
《国语·晋语》的记载,“黄帝之子二十五宗,其得姓者十四人,为十二姓,
姬、酉、祁、己、滕、箴、任、荀、僖、姞、儇、依是也”。在这十二姓中
没有张姓,这是不是在说挥不是黄帝之子而是黄帝之孙?如果我们认为古
籍中一些记载多少有些根据的话,最合理的解释是:挥是玄嚣之子,玄嚣
是得姬姓的黄帝之子。张先生的这份家谱,黄帝到挥差了玄嚣一世。

还有,挥得姓张,张良的父祖姓姬,张良怎么会是挥的后代?虽然挥本来
也姓姬,挥得姓张之后,子孙自然奉挥为始祖不会再姓姬了。说张良是黄
帝的姬姓后代还能勉强说得过去,不论说什么姓照古人的攀祖先的惯例,
都能说是黄帝的苗裔。鲁迅先生曾说过:“至于第三种,我没有看过
《清史》,不得而知,但据老例,则应说是爱新觉罗氏之先,原是轩辕黄
帝第几子之苗裔,遁于朔方,厚泽深仁,遂有天下,总而言之,咱们原是
一家子云。”较真起来,说张良是挥的后代于传统宗法一点也不相符。
再一个,从挥到张良的父祖中间70多世绝大多数也都是不可考的。

这份家谱拿在家里,供在张氏祠堂里,私底下炫耀炫耀也就算了,这样的
拿出来公之于众算不算是一个笑话?不知张先生几时染上了这古人攀祖先
的恶习,真替张先生不值。
