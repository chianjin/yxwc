\chapter[中秋]{中\qquad 秋}
\blfootnote{一九九六年。}

虽然,白天还是夏日般的闷热,夜晚寒意侵入衣底的时候,便觉着秋日已经到
来。书上常说的“金秋”,在合肥其实是看不到,夏日的浓绿不经意就变成了褐色,
无论如何也不能让人有一种愉悦的感觉。只有晴朗的天空中显现出的明净的蓝色,
倒还爽心悦目。行人的夏装,依旧潇洒;女孩的裙裾,依旧翩翩。只是商店门口巨
幅的月饼广告,大肆地宣告着秋天的到来,不肯让人多一些夏日的梦。

翻一翻日历,才知道中秋节没有几天了。一个人在异乡,节不节的无所谓,月
饼也是可有可无的。团圆只是在梦里的快乐。

月饼的来源,据说是元末农民起义时传递消息的一种有馅的圆饼。起义者们提
着竹篮,竹篮里放着刚刚烤出来夹着起义消息的圆饼,装着过节走亲戚的模样,向
关卒躬身哈腰地走过一个又一个关卡,心里却恨恨地说:“待得老子事成,杀你个
直娘贼。”至于以后如何有了团圆美满的象征意义,我是不大清楚的。想来大约是
由月圆到饼圆的罢,果然这样的话,始作俑者,怕的要是东坡了。如今,月饼好象
已不仅仅只是中秋节必备的美味点心了,正所谓是“吃的不买,买的不吃。”商店
里陈列在橱窗里的月饼包装得花花绿绿的,正如青楼的红尘女子待价而沽,一个要
卖上百;早知现在,月饼这玩意还是扔进水沟的好。几百块钱的月饼吃在嘴里,真
不知道是什么滋味。虽然,有月饼我还是要吃的,不是上百的也不是几十,两
三块钱已是不错的了。

在记忆中,最好吃的月饼是母亲做的芝麻糖馅糯米面饼。小的时候,家里不是
很宽裕,食品店里的月饼只能买上一两个孝敬奶奶。母亲想尽一切办法让我和妹妹
高兴。在农村的舅舅过年带来的一点芝麻,放在锅里炒得满屋飘香,再放在铁臼中
舂碎,拌上白糖,就是芝麻糖馅。我和妹妹早已馋得咽口水了。糯米面也是买一点
廉价的糯米,淘洗干净,泡上一天,也放在铁臼中舂细。比不得现如今商店中,塑
料包装袋中的水磨元宵粉。对于那时的一般家庭来说,已是一样奢侈品了。刚刚烤
好的芝麻糖馅的糯米饼我可以一下子吃三个。直到今天我也不能忘记,融化的糖汁
裹着芝麻粉流出来烫着我和妹妹的嘴,母亲脸上的苦涩的笑容。如今母亲再也不会
为月饼发愁。只是这月饼吃在口中,早已没有什么感觉了。

江南这个时候所习见的美味,都是水中生长的肥嫩的菱角、薏仁之类的。用长
长的树枝在塘中随意挑起,剥去外面的皮,含在嘴里凉凉的、甜甜的,慢慢地化作
满口的清香,没有一些些渣滓。琥珀色的封缸老酒更是“味轻花上露,色似洞中春。
”天上的圆月在淡淡的浮云中徜徉,印在水中,印在杯中;清凉的微风拂过水榭,
银色的光漾漾地散在水面。应该是最好的享受了。

在今天的日子里,又会有谁“举杯邀明月”。远离故乡的日子,中秋如同一个
又一个普通日子,平静而又平静。也许明年我该回家过一个中秋。
