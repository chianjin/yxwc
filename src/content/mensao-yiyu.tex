\chapter[呓语]{呓\qquad 语}
\blfootnote{二〇〇三年,短暂无聊赖的日子。}

\xsection{之一}

每天都告诉自己一定按时吃饭,每天都胡乱找点什么。

每天都告诉自己一定打扫卫生,每天都面对满屋子地垃圾。

每天都告诉自己一定要戒烟,每天都微笑地对着楼下杂货店老板说来两包。

每天都告诉自己一定12点睡觉,每天都要听到窗外马路对面卖早点的小摊子生火的声音才知道天就快亮了,该睡觉了。

\pskip
已经忘记早晨太阳如何升起。

已经忘记满天的繁星时什么样子。

已经忘记哪一张CD曾经是我的最爱。

已经忘记我来这里干什么又要去往何方。

\pskip
记忆中,只有黯淡的黄色和黑色。

昏黄的灯光,有气无力地照亮黑夜中破旧门楼的前一小片地方。门楼前的物件,还是当初的模样。

已经不是当初的模样,只有破旧摆在眼前。路边的窗户,亮着灯光,传来POP的节奏,徘徊在破旧门楼前面的旧巷。这不是雨巷,不会碰见撑着油纸伞丁香一样的姑娘。

旧的港口,还能停泊旧日的船只;旧的街巷,还有什么会路过此地。

\clearpage
\xsection{之二}

总是喜欢一个人看着微风中瑟瑟的无名的白色的小花;

总是喜欢一个人在微雨中徘徊在水边任柳枝摇摆在我的面庞;

总是喜欢一个人远远聆听古寺里传来的袅袅梵唱;

\pskip
不知道什么时候开始,我很享受寂寞。一个人的夜晚,翻检着落满灰尘的CD,翻检着泛黄的书页,桌上的清茶,冒着热气,外面的世界已经不再存在。

佛陀在灵鹫山上捻花微笑的时候,已经在梦中种下了毒药。去赞叹佛陀的智慧,还是去诅咒佛陀的恶毒?

曾经一个人,拖着行囊,走过虎跳峡峻峭山崖上的小路,峡谷中飘荡的风,吹着我的头发,蓝得不能再蓝得天空,总是有一种纵身而下的冲动,我也不知道我在寻找什么。

\pskip
窗外夜色已经浓重,我闻见马路对面卖早点的小摊子生火的木材燃烧的味道了,该去睡觉了。
又是一个享受寂寞的夜晚。

\xsection{之三}

醒来的时候,已经是下午时分。才是周五,周围很安静,只有公共汽车电脑报站器的声音:乘客您好,欢迎乘坐6路公共汽车,本车开往大兴镇。很清晰地钻进我的耳朵里。平时没有注意过的东西,此刻居然这样清晰实在,彷佛能触摸到它的影子。

\pskip
起来之后的第一件事情,摸索着烟盒,空了。于是靸了拖鞋,下楼,微笑着对着杂货店的老板,一如既往:来两包。点燃香烟,泡上一杯瓜片,慢慢咂着,我才算清醒。

\pskip
也许是清醒吧,也许还是在梦中。庄子不知道是庄子梦到蝴蝶,还是蝴蝶梦到庄子。我不是庄子,我也不知道庄子是否知道是庄子梦到蝴蝶,还是蝴蝶梦到庄子。我也不知道是否是什么怪样子的东西梦到的我。

\xsection{之四}

十二点已经过了半个小时,拿起手机又放下,放下手机又拿起。不知道是想拨一个电话,还是希望接到一个电话。记忆已经模糊不堪,梦里的旧巷淡如泛黄的照片。那扇窗户是否还亮着灯光。

\pskip
干白,在杯中就和我的记忆一样的泛黄。CD,在屋里就和我记忆一样的模糊。曾经水边的徘徊不再,只有酒和机器陪伴我的夜晚,今夜是否会有梦?不知道,是希望有梦,还是不希望有梦。没有梦的日子,远离爱恨,却是难耐的寂寞。有梦的日子,远离寂寞,却是理不清的爱恨。

\pskip
还是翻检翻检书吧,幽幽地吟了诗词,或许能不再思考这样到处都是正确答案的难题。或许是注定的寂寞,“梦里不知身是客,一晌贪欢”,一出口,就已然后悔。

\pskip
夏日无风的日子

该用某种东西来结束

这摇弋的渴望

\pskip
守望在杨柳岸边

于孤寂的夜晚

一任冷雨扑打窗棂

\pskip
月的朦胧

依然弥漫在这多情的季节

我却在寻找柳絮飘飞的家园

\xsection{之五}

天气又开始阴霾起来。我不晓得到底是否喜欢下雨。

\pskip
记忆里,总是回绕初春细雨中的柳色和渔歌。最爱江南细雨微蒙,独自漫步在河边的柳树下,任柳枝摇曳发端,任渔歌回荡耳畔,任雨丝吻在面庞。初春时刻虽是清冷,却也有无限的温柔。

\pskip
记忆里,总是回绕夏日暴雨中的呼喊和痛苦。暴雨中的呼喊只有两个人听见,暴雨中的痛苦只有两个人知晓。天地不应,人事尤远,世事无奈,“这次第,怎一个愁字了得”。

\pskip
而今,已然不再细雨漫步的闲趣,也不再有暴雨狂呼的痛苦,不再有年少的狂放。有的只是千篇一律的说辞,有的只是习惯而自然的职业般的微笑,有的只是所谓成熟的稳重。

\pskip
窗外传来孩子们奔跑的欢闹,生疏了,生疏了。
