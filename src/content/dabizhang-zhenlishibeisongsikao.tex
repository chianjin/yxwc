\chapter{针·历史·背诵·思考}
\blfootnote{* 2010年12月4日发表于《新语丝》。申明:本人对括号标记之外的全部文字完全负责,与方舟子以及新语丝网站无关。}
\blfootnote{* 本文网址 http://xys.org/xys/ebooks/others/science/dajia11/zhongyi2922.txt}

\subtitle{——致白衣闲士兼答张功耀}

\kaishu
〖空心括号中楷体为白衣闲士《〈天工开物〉与绣花针》
\footnote{http://xys.org/xys/ebooks/others/science/dajia11/zhongyi2918.txt}原文摘录。〗

\fangsong
【黑色括号中仿宋体为张功耀《从背诵历史到思考历史——答钱进先生》
\footnote{http://xys.org/xys/ebooks/others/science/dajia11/zhongyi2916.txt}原文摘录。】

\normalfont
诚然如白衣闲士所云
\kaishu
〖《新语丝》网站是以唯科学为特征的网站。我们质疑中医,质疑《本草纲目》,
质疑《黄帝内经》。但为什么迷信《天工开物》呢?〗
\normalfont
但是否因为古籍中或多或少存在这样那样的错误就全盘否定?是否因为宋应星
对蜜蜂描述的错误就怀疑其著作的全部?虽然中国古代的拉丝技术远远落后于
欧洲,但这不等于说中国古代没有拉丝技术。好在《天工开物》这部书是有
附图的,请白衣闲士看一看,看看古代中国人是怎么拉丝的。

不过呢,从张功耀的观点来看,白衣闲士“思考历史”的本领远远不如张功耀,
没有把“用铁尺一根,锥成线眼”象张功耀那样“思考”成
\fangsong
【制针之前,先把一块条型铁卷成一个圆筒,中间留一个小孔。】

\normalfont
这回本人真的荣幸之极,得以聆听张功耀教授的“教训”。原因在于本人
\fangsong
【在《新语丝》上发表文章,猛烈地批评我“瞎掰历史”。】
\normalfont
更因为
\fangsong
【这意思就是说,真实的历史在钱进那里,诸位如果要学习历史,听钱进的就行了。】
\normalfont
而且
\fangsong
【我从来不背诵现有的历史,尤其是那些按照“爱国主义模式”写出来的历史,
而钱进先生最擅长的恰恰就是“背诵历史”。】【因为我从来不“背诵历史”,
所以,我违背了钱进先生制定的天条,于是他要猛烈地咒骂我“瞎掰”。】

\normalfont
张教授实在抬举我了,说我是
\fangsong
【最擅长的恰恰就是“背诵历史”。】
\normalfont
真是谬赞,实在啊不敢当。从小上学起,小的历史分数就没有超过80分。
张教授更是抬举我要做玉皇大帝。不过小的姓“钱”,就是包括张教授在
内每天都少不了的东东。小的不姓玉皇大帝的那个“张”也就是张教授的
那个“张”,就算修炼百千万劫成了神仙也做不得玉皇大帝的储君,定不
了天条。

\fangsong
【我国什么时候发明制针技术,这本是可以自由讨论的。用不着像钱进
那样去“以背诵历史为依据”,对不愿意“背诵历史”的人大开骂戒。】
\normalfont
哦,小的终于明白了,张教授开恩“教训”小的这等
\fangsong
【最擅长的恰恰就是“背诵历史”】
\normalfont
顽劣之人的原因是小的对张教授
\fangsong
【大开骂戒。】
\normalfont
怕怕啊,“瞎掰”二字就是
\fangsong
【大开骂戒。】
\normalfont
赶紧翻翻中国社科院语言研究所《现代汉语词典(修订本)》生怕“瞎掰”真
的是开骂,看完解释,石头落地。上面的解释是:1.徒劳无益;白搭;2.瞎
扯乱讲。小的晕\~{}\~{}\~{}\~{}啊,晕菜的晕\~{}\~{}\~{}\~{}啊,实在
闹不明白“瞎掰”二字怎么就如同星爷嘴里哪个“什么mù”是
\fangsong
【大开骂戒】
\normalfont
鸟。

诚然,小的才疏学浅,不及教授学识渊博,不会“思考历史”只会“背诵历史”。
不过不会“思考历史”有时也是好的,不至于把“锥成线眼”“思考”成
\fangsong
【把一块条型铁卷成一个圆筒,中间留一个小孔。】
\normalfont
好在小的还可以把白衣贤士拉来陪绑接受张教授“教训”,不至于一个人度过
“寂寞的季节”。

小的在这里壮着胆子小声招呼一下,坛子里还有福建马尾人啊,有就吱一声。
敢请福建马尾的来做个见证。张教授“教训”道
\fangsong
【至于“绣花针”是什么针,以上钱进引出的这个文献实际上蕴含了答案:“凡
引线成衣与刺绣者,其质皆刚。惟马尾刺工为冠者,则用柳条软针”。依据这
个文献,福建马尾的人做刺绣,用的是“柳条针”。这个“柳条针”其实不是金
属的,是木质的(如竹子做成的针),有的地方甚至用鱼刺。】
\normalfont
“其质皆刚”,刚者,钢乎?福建马尾人用的“柳条针”是木制或竹制的?学医的
童鞋应该都知道,医生的手术刀有一种称呼叫做“柳叶刀”,敢问一声,这“柳
叶刀”是柳树叶子做的吗?“凡引线成衣与刺绣者,其质皆刚。惟马尾刺工为冠
者,则用柳条软针。”这句简单不过,大凡引线做衣服和刺绣的针,质地刚硬。
只有马尾的刺绣工匠师最优秀的,用的是如柳树枝一样软的针。这就是所谓的
柳条软针。张教授“思考历史”的结果如此,小的佩服得五体投地,滔滔江水……,
黄河泛滥……

张教授接着“教训”道
\fangsong
【钱进还引出文献证明,我国早在战国的时候就有刺绣。他以为那战国时期的
刺绣就是金属针绣制出来的。其实,战国的刺绣不是用金属针绣制的。我国先
民最早用的针是骨针(包括鱼刺)。商代生产的某些玉器,镂空出了许多比现
代绣花针还小的孔(河南博物馆有这样的文物)。它们也不是用金属针加工出
来的,而是用骨针加工出来的。知道了这一点,就可以理解,为什么我国古代
金属制针业很落后,而刺绣却很精致。】
\normalfont
张教授“教训”非常正确,小的的确以为中国古代的刺绣都是牙咬出来的。直到
小的看到了中国历史博物馆藏有北宋济南刘家功夫针铺印刷广告铜板,这块铜
板下方文字为“收买上等钢条造功夫细针不误宅院使用客转与贩别有加饶请记
白”,才知道最迟宋代可能就已经使用拉丝技术制作针。

张教授宣称
\fangsong
【我们为了突出主题,必须省略掉某些无关紧要的细节。】
\normalfont
对啊,对啊,这“必须”,这“省略”,就“必须省略”成了
\fangsong
【我国古代不会生产金属针。】

\normalfont
张教授
\fangsong
【坦率地说,钱进提出的问题没有什么回答价值。只是因为《新语丝》的读者
很广,为正视听,才勉强写了这么几句话。】
\normalfont
哦,看来我自作多情鸟,张教授不是开恩教训
\fangsong
【最擅长的恰恰就是“背诵历史”】
\normalfont
的小的,而是
\fangsong
【为正视听,才勉强写了这么几句话。】
\normalfont
罪过,罪过,为了小的没有什么价值的问题耗费了张教授宝贵的脑细胞,损
失了张教授宝贵的时间来
\fangsong
【思考历史】。

\normalfont
教授自然
\fangsong
【不会接受任何形式的纠缠】
\normalfont
,而小的不是教授自然会纠缠任何“瞎掰”。
