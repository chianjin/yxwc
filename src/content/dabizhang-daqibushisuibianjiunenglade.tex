\chapter{大旗不是随便就能拉的}
\blfootnote{* 2006年10月11日发表于《新语丝》。申明:本人对本文全部文字完全负责,与方舟子以及新语丝网站无关。}
\blfootnote{* 本文网址 http://xys.org/xys/ebooks/others/science/dajia7/zhongyi156.txt}
\blfootnote{* 更正网址 http://xys.org/xys/ebooks/others/science/dajia7/zhongyi179.txt}

\subtitle{——读张功耀《孔子不信巫医考》}

我前面的文章《读张功耀先生《关于废除阴历的新设想》》 中说了“北大科学
史与科学哲学”网站上在这篇文章成文的时候时候有18篇张先生的文章。上篇
文章主要是为了张先生关于“废除阴历”的主张,其他并未在意,这两天又看了
张先生《孔子不信巫医考》\footnote{http://hps.phil.pku.edu.cn/viewarticle.php?sid=1994}
一文。严重怀疑张先生的古文水平,我以为张先生如果有自知之明的话,在
未弄清楚一些文献的意思之前,在未全面了解孔子的观点之前,不要把孔子
拉来作为“废除中医”的一面旗帜。深为张先生不值。

张先生文中说“孔子一生笃守‘不语怪、力、乱、神’(《论语·述而》)的治
学格言。正因为如此,他对那些不明不白的巫和医是从来不相信的。”前一
句意思大体不错,但表述很成问题。“子不语怪、力、乱、神”这句话是孔
子弟子或者再传弟子以自己的口气记述下来。孔子一生没有明确表述过自
己“不语怪、力、乱、神”,而且孔子之前从未存在过“不语怪、力、乱、神”
成文说法,“格言”二字从何谈起?张先生把个前后顺序搞反了。后一句,
张先生显然自己设定了一个小前提:孔子认为医属于怪、力、乱、神,于
是张先生得到了结论:“正因为如此,他对那些不明不白的巫和医是从来
不相信的。”一个典型的三段论。

大提前:孔子一生笃守‘不语怪、力、乱、神’(《论语·述而》)的治学格言。

小前提:孔子认为巫和医属于怪、力、乱、神。

结\quad 论:正因为如此,他对那些不明不白的巫和医是从来不相信的。

乍看起来,很有说服力。仔细看来,问题仍然存在:

1.  “子不语怪、力、乱、神”,这句话意思很明显:孔子从来不谈论怪、
力、乱、神的一类东西。“不谈论”是否等于“不相信”?请张先生教我。

2.  张先生省略的小前提:孔子认为巫和医属于怪、力、乱、神。也没
有明确的文献证据,也不是板上钉钉的事情。

这个三段论,因两个前提存在漏洞,故结论是不能成立的。这个问题先
丢一边,张先生接下来,列举文献中的一些记载来支持自己“他(孔子)
对那些不明不白的巫和医是从来不相信的”的结论。

张先生文中用来支持“孔子不信巫的行为”的例子,《论语·述而》:子
疾病,子路请祷。子曰:“有诸?”子路对曰:“有之。诔曰:祷尔于上
下神祗。”丘曰:“丘之祷久矣。”张先生行文:“此处的‘诔’,原是一
种周朝旧制,它专为士大夫以上的官员死后祈求悔过还善、神灵护佑
而制作的祷文。在民间,‘诔’则是对任何无理(说不清理由)事物的
求福祈祷。”认为诔是周代制度大体上也还说得过去,不细论了。以
白话文来表述这段记述是这样的:孔子得了重病,子路请求(向神祗)
祈祷。孔子问:“有这样的事吗?”子路回答:“有这回事,诔文说:
‘替你向天地神祗祷告。’”孔子说:“我孔丘(象你所说)的祷告已经
有很长时间了。”“丘之祷久矣。”到底什么含义?这很重要,关系到
这段描述能否用来支持张先生“孔子不信巫的行为”。字面上解释为
“我孔丘(象你所说)的祷告已经有很长时间了。”应该不错。那么
请张先生教我,您是如何从这段记述得到“孔子不信巫”的?
《尚书·金縢》记载了周公因为武王生病而向先公先王祈祷的事情:
武王有疾,不豫……周公于是乃自以为质,设三坛,周公北面立,戴
璧秉圭,告于太王、王季、文王。史策祝曰:“惟尔元孙王发,勤
劳阻疾。若尔三王是有负子之责于天,以旦代王发之身。旦巧能,
多才多艺,能事鬼神……”(尚书今文古文之分,古文尚书早就被证
明有伪,而今文尚书不伪。《金縢》这篇今文尚书中有。)孔子一
生以周公为榜样,周公为武王生病而祈祷,孔子怎么会反对祈祷这
事?这个例证不能用来支持张先生“孔子不信巫”的观点。

张先生说:“人们便把实现好转的希望寄托于祷念这种‘诔文’”又
弄混了。人们是把希望寄托于祈祷神祗的行动,祷念诔文只是这
一行动中一个环节而已。

指出张先生一个错误,张文:“有文献表明,通常它以诔其患者
的功德来祈祷。”《周礼·春官·大祝》:作六辞以通上下、亲疎、
远近,六曰诔。《墨子·鲁问》:诔者,道死人之志也。
《说文》:诔,谥也。从言耒声。可见原本诔大体相当现在的
悼词。中国古人的观念里,祖先死后也成神(祖先神,这是中
国传统文化的一个特色)。这样一来,诔用于死后成神的祖先
也是可以。《尚书·金縢》记载透露一个消息,古人为病祈祷
于祖先神。虽然后来佛教传入、道教兴起之后,又转而祈祷于
宗教神祗,而祈祷祖先的遗迹仍然可循。我小时候得病的时候,
我奶奶就会说求我死去的爷爷保佑我。张先生这个说法,不知
道是从那些文献得到。

再看看张先生用来支持“孔子不信医”的例证,《论语·乡党》:
“康子馈药,拜而受之曰:‘丘未达,不敢尝’。”这段话的含义
是什么?来看看辜鸿铭如何用英文描述这件事情的:
\begin{quote}
On one occasion when a noble, who was the minister in
power in his native State, sent him a present of some
medicines, he received it respectfully, but said to
the messenger: `Tell your Master I do not know the
nature of the drugs: therefore I shall be afraid to
use it.'
\end{quote}
辜鸿铭对孔子“丘未达,不敢尝”的理解应该是这样的:孔子
说我不知道这药的药性,所以我不敢用。再看看理雅各的英
文译文:
\begin{quote}
Chi K'ang having sent him a present of physic,
he bowed and received it, saying,`I do not know it.
I dare not taste it.'
\end{quote}
文有繁省,意则相同。按照这两人对这段描述的理解,反过
来理解孔子的话,是否也可以这样,如果知道药性的,孔子
也是会服用的。我现在不能确定孔子是否有这层意思,这个
可以继续讨论。但是我要提出对这段话的另一个理解。辜鸿
铭、理雅各都把“丘未达,不敢尝”六个字中的“达”解释为知
道、了解的意思不能说错,达的确有这层意思。“达”还有另
一次意思。《孟子·尽心上》:穷则独善其身,达则兼善天
下。这里“达”是发达,当时来说就是出仕做官。这六个字就
可以这样解释:我孔丘没有官职,不敢服用(您这位首辅大
臣送来的)药。孔子一生为官连头带尾五年,鲁定公九年
(公元前501年)出仕中都宰,次年升至司寇,定公十三年
(公元前497年)因季桓子接受女乐迷恋歌舞不理朝政,孔
子去鲁。此后孔子周游列国,希望遇到明主出仕为官一展
政治才能,虽然不少诸侯都非常礼遇孔子,但没有一个诸
侯肯用他。就这样周游列国14年。直到哀公十一年(公元
前484年)才回到鲁国。是年齐伐鲁,孔子弟子冉有帅鲁师
与齐战获胜。季康子问冉有指挥才能从何而来?冉有答曰
“学之于孔子”。季康子派人以币迎孔子归鲁。孔子回到了
鲁国,以为获得了重新出仕的机会,当时位于群卿之首的
季康子会重用自己。孔子回到鲁国之后,季康子很礼遇孔
子,不乏嘘寒问暖,“馈药”就是一证;但根本没有请孔子
出仕的意思。孔子只好删定《诗》、《书》,编撰《春秋》。
结合当时的孔子的境遇和礼法,也认为孔子这六个字的回
答是提醒季康子:你请我回鲁,就该用我为官;你是首辅,
我现在是布衣,官民不能越礼,我不敢服用你馈赠的药。
这样的解释也合乎情理。虽然,我不敢说这种理解就是正
确的,但不管是从哪个方面来理解孔子的回答,我怎么也
看不出来如何就能确证“孔子不信医”。

接下来,张先生引述了当时一件大事,来说明孔子对于医的
态度。张文:“孔子在《春秋·昭公19年》中记载过一件大事。
这年夏,鲁国许悼公患了疟疾。5月,饮了其子许止给的中药
后死亡。然后,许止奔到晋国写书自责,并告诫后人说:‘尽
心力以事君,舍药物可也。’(《左传·昭公19年》)”。张先
生古文阅读水平我真真怀疑到底。《春秋·昭公十九年》原文:
十有九年春,宋公伐邾。夏五月戊辰,许世子止弑其君买。
己卯,地震。秋,齐高发帅师伐莒。冬,葬许悼公。“鲁国许
悼公”,稍有一些古文知识或者历史知识就要笑掉大牙。许悼
公是许国的国君,“鲁国许悼公”何许人也?孔子以他的春秋
笔法记载的是:夏五月戊辰,许世子止弑其君买。读春秋不看
传是不行的,《左传·昭公十九年》这样记述:夏,许悼公疟。
五月戊辰,饮大子止之药卒。大子奔晋。书曰:“弑其君。”君
子曰:“尽心力以事君,舍药物可也。”《左传》的记述没有张
先生说说的其子止写书自责,更没有说止告诫后人如何如何。
先看一个问题,许悼公的世子止姓什么?《汉书·地理志上》
颍川郡“许”下原注云:故国,姜姓,四岳后,太叔所封,二十
四世为楚所灭。许慎在《说文·叙》中自述先世则说:大岳佐夏,
吕叔作藩,俾侯于许,世祚遗灵。《说文·邑部》云:鄦,炎帝
太岳之胤甫侯所封。又《左传》隐公十一年孔疏引杜预《春秋
释例·世族谱》云:许,姜姓,与齐同祖,尧四岳伯夷之后也。
周武王封其苗裔文叔于许。周武王封的这个许国是姜姓。许慎
也承认自己是大岳之后。许国贵族何时以许为姓而不姓姜了呢?
《姓纂》:姜姓,炎帝之子孙,周武王封其裔孙文叔于许;后
为楚所灭,子孙分散,以国为氏。原来许公族后裔姓许是许灭
国之后的事情了。那么许悼公世子止便不姓而姓姜。原来不是
许止,是姜止。再来,张文:“许止奔到晋国写书自责,并告诫
后人说:‘尽心力以事君,舍药物可也。’”完全乱解典籍。
《左传》:书曰:“弑其君。”这里的“书曰”是“书上讲”或者书
作动词讲“写为”,“君子曰”就是君子说,不是许国国君之子。
张先生所谓止自责并告诫后人,就是根本没有的事情。孔子之
所以把许世子止进药给许悼公,许悼公服药后而亡的事情记为
“许世子止弑其君买”实际上跟医没有多大关系。《左传》上说:
君子曰:“尽心力以事君,舍药物可也。”意思是说:尽心尽力
(按照礼法的要求)服事国君就行了,用不着你去进什么药物。
比较隐晦,作为解释《春秋》的三传其他两传说得比较明白。
《谷梁传》:许世子不知尝药,累及许君也。还是有点不大明
白。《公羊传》的意思就比较直白了:止进药而药杀也。止进
药而药杀,则曷为加弑焉尔?讥子道之不尽也。意思说:止进
药害死了他爹,那是药杀(不是止杀的),孔子为何要加止
“弑君”的罪名?那是为了讥刺他不尝药未尽孝道。重点不是说
药不行,而是说止不合礼法。张先生接着说:“当时的人们对
进药所产生的过失是看得很严重的。”张先生或者没有搞清楚孔
子的真实含义,有意或无意歪曲孔子的意思。正如《春秋》记:
赵盾弑君案。明明不是赵盾指使,也不是赵盾动手的,《春秋》
这么记,是说赵盾没有尽到自己作为臣子的责任,所以这笔帐
要算在你赵盾的头上。《春秋》是孔子宣扬他政治主张的一部
著作,孟子很理解孔子作《春秋》的心思,《孟子·滕文公下》:
孔子成《春秋》,而乱臣贼子惧。孔子关心的是政治礼法,哪
里有多少嫌功夫去讨论医的问题。张先生接着又误解或者曲解
杜预一番:“据《春秋三传集注》的作者杜预分析,孔子写
《春秋》遵守褒贬分明的原则。孔子对臣子向君王进药,无论
是否当场毒死,也无论进药前是否由臣子尝过,都是当作罪恶
看的。”来看看张先生文中引用的杜预的注:盖以悼公之死,
由于世子之药。则止虽非弑,而弑君之罪,止有不得而辞者。
故加弑焉。所以教天下之为臣子者也。怎么理解?我这样理解:
悼公因为世子的药而死。止虽然(主观上)不是弑君,但是
(按照礼法)止不能摆脱弑君的罪名,所以记为弑君,用来教
训天下那些做臣子的。无论如何看不出“孔子对臣子向君王
‘进药’的禁止程度是毫不含糊的。”更遑论张先生说:“这也反
映出孔子既不信巫也不信医的坚决性。”

张先生接下说:“我们就可以从一个新的角度来理解孔子的另
外一句格言:‘人而无恒,不可以作巫医。’(《论语·子路》)”
不忙看张先生如何解释,现看看《论语·子路》原文,子曰:
“南人有言曰:‘人而无恒,不可以作巫医。’善夫!不恒其德
或承之羞。”子曰:“不占而已矣。”哦,不是孔子的格言,而
是孔子引用“南人”的说法。孔子自称“述而不作”,引用格言加
以阐述是孔子表明自己主张的一贯的方法。张先生把这话说是
孔子的格言,或许用来进一步提高孔子的光辉形象。如果是这
样就不必了,孔子不会因为某个人抬高而增一分光辉,更不会
因为某个人贬低而减一分光辉。类似的说法,《礼记·缁衣》
也有,子曰:南人有言曰:“人而无恒,不可能为卜筮。”考古
发现的郭店楚简《缁衣》作:宋人有曰:“人而无恒,不可以为
卜筮。”为什么《缁衣》中记述为“卜筮”,而论语中记述为“巫医”?
我没搞清楚,暂且不论吧,或许就是当时巫医不分的真实反应。

张先生把“人而无恒,不可以作巫医”作一番自己的理解:“笔
者以为,孔子这里的‘恒’,是指一种恒定不移的因果规律。既
然医和巫不能确定它们与疾病之间恒定不移的因果关系,则所
有的人都没有权力为病人施行巫和医。因此,这句话的正确解
释应该是:‘人(生老病死)是无常的,不可以为人施行巫术
或医术。’这就是中国历史上经由孔子表达出来的也许是最早
的‘废医论’观点”。张先生的理解能否成立,先放一下。

说起卜筮来,孔子跟卜筮很有关系。孔子先人是宋人,宋人是
商贵族后裔在周的封国,孔子是商贵族后代是无疑的。商人是
有名的卜筮高手,武王灭周以后还特地请教箕子关于卜筮的东
西。记载于《尚书·洪范》,整篇基本上都是箕子用《易》、
五行等等卜筮概念来解释天地伦常。孔子与《易》以及《易传》
的关系,历来争论纷杂。最早见于《史記·孔子世家》:孔子
晚而喜《易》,……读易,韦编三绝,曰:“假我数年,若是,
我于易则彬彬矣”,韦编三绝的成语出于此。但是《论语》没见
到明确记述,也不知道太史公从何而得到这样记述的。近年大
量帛书和竹简出土之后,增加了不少孔子与易的记述,但并未
解决这个问题,争论依然激烈,不过有一点应该是没有疑问的:
孔子所主张的礼当中,卜筮是许多礼仪的一个不可缺少的步骤。
《仪礼·士冠礼》开篇即曰:“筮于庙门。”

再论“人而无恒,不可以作巫医”怎么解释。关键要把“恒”字的
意思解释清楚。《说文》:恒,常也。从心从舟,在二之閒上
下。心以舟施,恆也。古文恆从月。《詩》曰:“如月之恒。”
《易·序卦传》:恒者,久也。《易·系辞下传》:恒,德之固
也。《诗·小雅·小明》:无恒安处。《礼记·月令》:文绣有
恒。疏:恒,故也。必因循故法也。《周禮·夏官·司弓矢》恒
矢痺矢,用诸散射。注:恒矢。安居之矢也。可见“恒”字的意
义非常明确。张先生生拉硬扯,说这里的“恒”是“一种恒定不移
的因果规律”,古人把张先生用来解释典籍的方法叫做“增字解
经”实不可取。退一步,就按照张先生“增字解经”的方法来看看,
“人而无恒”怎么解释。语法结构来看:人是主语,而是虚字,
无是谓语,恒是宾语。那么这四个字按照张先生关于“恒”的解
释意思应该是:人没有一种恒定不移的因果规律,勉强说的过去。
再看后一句“不可以作巫医”,主语是什么?显然承上句省略了
主语,主语还是人,作是谓语,巫医是宾语。这句意思是:
(人)不能研究(学习、施行)巫医。当时,汉语作的意思不
能解释为做,“述而不作”的作当著书立说讲,虽然《史记·孔
子世家》说:乃因史记作春秋,上至隐公,下讫哀公十四年,
十二公。这个“作”只能当编撰讲。“作巫医”的“作”应当当作研
究、学习或者施行的意思讲。好了,按照张先生的对恒的解释,
把前后句连起来看:人没有一种恒定不移的因果规律,不能研
究(学习、施行)巫医。这一连就有点前言不搭后语了。或许
张先生也是觉得别扭,于是强解释为“既然医和巫不能确定它们
与疾病之间恒定不移的因果关系,则所有的人都没有权力为病
人施行巫和医。”前面一句主语都换调了变成了巫和医,跟原
意差了几万里,还有这里的“它们”二字指谁?人吗?后一句里
的“没有权力”和“为病人”又是“增字解经”,严格起来,后一句
只对了三分二“施行巫医”,“所有人”也有“增字解经”的嫌疑。
张先生可能觉得还是解释有点别扭吧,接着又把自己的解释概
括了一下:“人(生老病死)是无常的,不可以为人施行巫术或
医术。”哦,这里“恒”的意思又回到了本来意义上了。“生老病
死”四字还是“增字解经”不谈了。“是无常”又漏了窟窿,“人而
无恒”的“无”是谓语,到了张先生这里跟后面的“恒”连到一起变
成了宾语。这句话其实没什么难以理解的地方,看看《论语》
中这个说法完整记述,后面有“善夫,不恒其德或承之羞。”这
句话出自《易·恒》,孔子引述来作为“人而无恒,不可以作巫
医。”的评价,这很是符合孔子“述而不作”的一贯做法。“好啊,
《易经》上讲:不能恒定德行就会蒙受耻辱。”回过头来看“人
而无恒,不可以作巫医”,恰当的解释是:人如果没有恒定不变
的德行,就不能不能研究(学习、施行)巫和医。不必如张先
生那样拐弯抹角,解释得通通顺顺。张先生牵强的解释,我高
中时代的语文老师马先生是不会给及格的。

张先生拿不是孔子的话来支持“这就是中国历史上经由孔子表
达出来的也许是最早的‘废医论’观点”,且又是曲解。我不晓
得这算是“厚诬古人”还是应该算是“厚溢古人”?

张先生既不能证明“孔子不相信巫”,也不能证明“孔子不相信
医”,就赶紧下结论:“孔子既不相信祷告,也不相信医药,
病却照样好了,而且活了70多岁。”

张先生说:“孔子有三慎:‘齐、战、疾’”。战、疾好懂。“齐”
啥意思?查《论语·述而》原文:子之所慎。斋、战、疾。哈
哈,原来是“斋”!由此想到《笑林广记》的一则故事《斋戒库》,
信手录来以为一笑:一监生姓齐,家资甚富,但不识字。一日,
府尊出票,取鸡二只,兔一只。皂亦不识票中字,央齐监生看。
生曰:“讨鸡二只,免一只。”皂只买一鸡回话。太守怒曰:
“票上取鸡二只,兔一只,为何只缴一鸡?”皂以监生事禀,
太守遂拘监生来问。时太守适有公干,暂将监生收入斋戒库内
候究。生入库,见碑上“斋戒”二字,认做他父亲“齐成”姓名,
张目惊诧,呜咽不止。人问何故,答曰:“先人灵座,何人设
建在此?睹物伤情,焉得不哭。”但愿不是张先生误读,只是
笔误而已。后面张先生根据他自己以为确证的孔子“不信巫也
不信医”说:“不管是他自己,还是他的弟子病了,他既不主张
做祷告,也不主张请医生。”前面已经讲了,张先生并没有确
证,这话并不能就这样成立了。

张先生有引用《论语》中的一些记述,要说明:“孔子特别重
视日常的保健工作。《论语》中记载有许多孔子谈论树立正确
的饮食起居习惯的格言,至今还颇有劝导价值:‘食不厌精,
脍不厌细。食饐而餲,鱼馁而肉败,不食。色恶,不食。臭恶,
不食。失饪,不食。不时,不食。割不正,不食。不得其酱,
不食。肉虽多,不使胜食气。惟酒无量,不及乱。沽酒市脯,
不食。不撤姜食,不多食。’又:‘菲饮食而致孝乎鬼神,恶衣
服而致美乎黻冕,卑宫室而尽力乎沟洫。’”。前面几句“食不
厌精,脍不厌细”等等,与其说孔子重视饮食保健,毋宁说孔
子要坚持他的礼法。比如“割不正,不食。”肉切得不规矩,
孔老夫子也不吃,这跟饮食保健应该没有多大关系吧。《礼记》
记载了许许多多不同场合下应当遵守的饮食法则,甚至规定了
不同种类的菜肴放置的位置,显然不是为了饮食保健,而是为
了礼法。当然孔子的某些说法大体上也合乎现代饮食保健的要
求,但“食不厌精,脍不厌细”这两句却不能符合现代饮食保健
的要求。后面张文中“菲饮食而致孝乎鬼神……”见于《论语·泰问》,
子曰:禹,吾无间然矣。菲饮食,而致孝乎鬼神;恶衣服,而
致美乎黻冕;卑宫室,而尽力乎沟洫。禹,吾无间然矣。看看
通常的解释:禹啊,我真是没什么可说的了。他吃的饭食非常
简单,但祭祀祖先和神明却十分丰盛。他平时穿的衣服很破旧,
但仪式上的礼服帽子却极为讲究。他住在低矮的宫室里,整天
在外面尽力修治沟渠水道。大禹啊,我们真的无法再形容他了。
“孝乎鬼神”,对鬼神很孝顺,怎么孝顺鬼神,当然是用丰盛祭
品来祭祀鬼神了。黻,仪式上穿的礼服;冕,仪式上戴的帽子,
“美乎黻冕”意思显然是说仪式上礼服帽子非常精美。孔子这段
话显然是赞叹大禹虽然日常生活简陋,对祭祀鬼神、举行仪式、
关心民众疾苦一点也不马虎。这跟孔子的起居保健有什么关系?
更值得注意的是,孔子赞美大禹祭祀鬼神一点都不马虎,是不
是说明孔子很相信祝祷?

张先生胡乱解释一番经典作为自己的论据,下了一个小结论:
“于是,就有‘圣人不治已病治未病’的格言流传至今了。”这个
格言来源于何?源自《素问·四气调神大论》,原文:是故圣人
不治已病治未病,不治已乱治未乱,此之谓也。嘿嘿,张先生
的麻烦大了。《素问》是《黄帝内经》的一部分,《黄帝内经》
又是中医理论的基本经典。引述《黄帝内经》的文字来抬举孔
子反对中医。张先生,神人也。

张问最后一段说:“遗憾的是,孔子的这个‘废医论’格言被后
来的一些中医理论家歪曲了”各位看了我的文章,估计应该已
经明白了到底是谁在歪曲孔子。张文紧接着说:“孔子‘不治已
病’”这根本不是孔子说的,而是中医典籍《黄帝内经》的说法。
张文最后说:“不过,我们有理由相信,对于告别了‘怪、乱、
力、神’,医理和病理都说得十分明确,甚至还有实验证明的‘医’,
孔子还是会接受的。”张先生又在混淆视听,“不语”变成了张
先生的“告别”。虽然我也有理由相信,如果孔子知道药性,他
老夫子一定还会接受。我也有理由相信,如果孔子还在世上,
估计要给张先生七八个耳光的。不过也不一定,《笑林广记》
还有一则《证孔子》说:两道学先生议论不合,各自诧真道学
而互诋为假,久之不决。乃请证于孔子,孔子下阶,鞠躬致敬
而言曰:“吾道甚大,何必相同。二位老先生皆真正道学,丘
素所钦仰,岂有伪哉。”两人各大喜而退。弟子曰“夫子何谀之
甚也!”孔子曰:“此辈人哄得他动身就可了,惹他怎么?”又一笑。

从张先生此文的题目《孔子不信巫医考》,想到了另一个人的
文章题目《孔子改制考》,张先生不是康有为,没有康有为的
考据功底,何苦来弄出一篇混身是洞的文章?康有为被人称为
“康圣人”,张先生是否也想得个“张圣人”的名号。可惜就是
“康圣人”也没有圣过几个年头。

说得严重一点,张先生的这篇文章就是现在大家深恶痛绝的垃
圾文章之一种类型。预先定了自己的观点,生拉硬套胡乱曲解
解释典籍来凑合自己的观点,拉古人作大旗。正如我文章的题
目,大旗不是随便就能拉的,连最起码的古文阅读能力都值得
怀疑,又来凑什么热闹?张先生是既然是搞自然辩证法(新近
又出了一个新名字:“科学技术学”还有书:《科学技术学导论》
\footnote{http://school.freekaoyan.com/hunan/csu/zhuanye/20060607/10193.shtml}
(自然辩证法概论)张功耀主编,中南大学出版社,网上还有
说是:《科学技术学悖论》\footnote{http://www.timesinfor.com/book214952.html}
(自然辩证法概论)张功耀--21世纪研究生课程教材,没见着
原书,不知谁是真的。不管真假至少多了一个学科,又有一批
人有了饭碗,功德无量也)、科学哲学出身的,你就在你自己
熟悉的范围内去讨论“废除中医”有什么不好,可惜张功耀先生
未必够听得进:在自己熟悉的领域踏踏实实做点工作是应当的,
在自己不熟悉的领域小小心心看点东西是要紧的。
