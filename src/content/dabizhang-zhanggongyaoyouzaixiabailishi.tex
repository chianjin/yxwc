\chapter{张功耀又在瞎掰历史}

\blfootnote{* 2010年11月30日发表于《新语丝》。申明:本人对本文全部文字完全负责,与方舟子以及新语丝网站无关。}
\blfootnote{* 本文网址 http://www.xys.org/xys/ebooks/others/science/dajia11/zhongyi2915.txt}

张功耀新文《说说京剧和针灸这一对难兄难弟》
\footnote{http://www.xys.org/xys/ebooks/others/science/dajia11/zhongyi2909.txt}
中如何评价京剧和针灸我不管,说实话对二者我都没有什么兴趣。不过呢,我对历
史比较有兴趣。看了张功耀这篇大作,哑然——张功耀又在瞎掰历史。

张功耀在文中说:“我国古代不会生产金属针。像现在这样精细的针灸用针,也是
晚清的时候从西方引进来的。李白‘铁棒磨成针’的故事说明,那时中国的制铁业只
能生产‘铁棒’和粗大的‘铁钉’,不可能生产类似于‘绣花针’的针灸用针。就是《黄
帝内经·灵枢》所记载的‘九针’,最锋利的也只是‘锋如黍粟之锐’。其中,完全没
有穿透皮肤和肌肉进针的针灸记载。穿透皮肤深入到肌肉或骨头表面的进针,是后
人望文生义发挥出来的。至于现代中医的‘电针疗法’,无论在古代医术,还是在现
代医术中,都毫无依据。”用“铁棒磨成针”的故事来论证“那时中国的制铁业只能生
产‘铁棒’和粗大的‘铁钉’,不可能生产类似于‘绣花针’的针灸用针。”这一结论,真
乃高人也!

看一段明朝人宋应星的记载,在其《天工开物·锤锻·针》条中这样说:“凡针先锤
铁为细条。用铁尺一根,锥成线眼,抽过条铁成线,逐寸剪断为针。先鎈其末成颖,
用小槌敲扁其本,钢锥穿鼻,复鎈其外。然后入釜,慢火炒熬。炒后以土末入松木
火矢,豆豉三物罨盖,下用火蒸。留针二三口插于其外,以试火候。其外针入手捻
成粉碎,则其下针火候皆足。然后开封,入水健之。凡引线成衣与刺绣者,其质皆
刚。惟马尾刺工为冠者,则用柳条软针。分别之妙,在于水火健法云。”宋应星生
于1587年卒于1666年,明亡于1644年。他的记载至少比晚清时期早了200多年,而
且这种工艺显然是先锻、再拉丝、最后淬火,是非常成熟的制针工艺,在此之前必
然有上千年的发展。而现代制针除了设备先进以外,其工艺更加精细以外,其步骤
是基本一致的。

再看中国刺绣的历史,虽然“舜令禹刺五彩绣”不一定可信,但早期的刺绣遗物显示:
周代尚属简单粗糙;战国渐趋工致精美,湖北江陵马山硅厂一号战国楚墓出土的绣品,
这标志此时的刺绣工艺已发展到相当成熟阶段。汉代王充《论衡》记有“齐郡世刺绣,
恒女无不能”,足以说明当时刺绣技艺和生产的普及,最具代表性的是湖南长沙马王
堆汉墓出土的刺绣残片。到了宋代,更是刺绣发达高峰时期,特别是开创纯审美的
艺术绣,《宋史·职官志》“宫中文绣院掌纂绣”,徽宗年间又设绣画专科。此后明清
两代更是在宋人优良刺绣的基础上,成为我国历史上刺绣流行风气最盛的时期。我想
请张先生告我,“工欲善其事,必先利其器。”如果中国古代“制铁业只能生产‘铁棒’
和粗大的‘铁钉’”,那么中国古代的刺绣都是用牙咬出来的吗?

张功耀先生最好不要谈及历史,以前在新语丝的辩论看来是忘记了。总拿自己的软
肋给别人戳,真的很意思吗?
