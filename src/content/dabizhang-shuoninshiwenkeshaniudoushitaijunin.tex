\chapter{说您是“文科傻妞”都抬举了您}
\blfootnote{*2007年7月11日发表于《新语丝》。申明:本人对本文全部文字负责,与方舟子以及新语丝网站无关。}
\blfootnote{*本文网址http://xys.org/xys/ebooks/others/science/dajia8/zhongyi1093.txt}

\subtitle{——评张功耀《“治未病”究竟治什么?》}

jhuang在新语丝发文\footnote{http://xys.org/xys/ebooks/others/science/dajia8/zhongyi1037.txt}
说张功耀信口胡言,引得张先生雷霆大怒\footnote{http://xys.org/xys/ebooks/others/science/dajia8/zhongyi1041.txt}。
三天之后,张先生在自己的博客发表了《“治未病”究竟治什么?》\footnote{http://zhgybk.blog.hexun.com/10433529\_d.html}
(以下简称《治》文)。笔者曾建议张先生尽管可以主张“告别中医”,最好还是不
要谈历史!尽管之前,笔者对张先生的历史文献功夫产生过强烈怀疑
\footnote{http://xys.org/xys/ebooks/others/science/dajia7/zhongyi156.txt},
可惜笔者毕竟人微言轻,在张先生根本当不得一回事。读了张先生的《治》文,真
正要说:这不是信口胡言又是什么?

张先生《治》文中说:“众所周知,孔子不相信中医的医术或巫术具有治病的功能。”好一
个“众所周知”,请教张先生,到底要多少人知道才是“众所周知”?十分之一?那也要一亿多
人!假如真的如张先生所说“众所周知”了,又如何有必要张先生特特撰出一篇《孔子不信巫医
考》\footnote{http://hps.phil.pku.edu.cn/viewarticle.php?sid=1994}这样的浑身是洞
的文章?难道张先生真的以为您一篇文章就真的能“众所周知”了。笔者
曾撰文讨论过张先生所用的历史文献根本不能证明“孔子不信巫医”,正如笔者所讲,笔者人
微言轻,张先生当然可以不屑一顾。不过,张先生最好还是顾一下,这个“不顾”的后果,那就
是:说您张功耀先生是“文科傻妞”都抬举了您!“文科傻妞”还不至于通篇乱解文献。

张先生《治》文中又提“孔子拒绝子路请祷、不尝季康子馈药、止进药许悼公致死”三件事,
笔者已经讨论过这三件事情,根本不能证明张先生所谓“孔子不相信中医的医术或巫术具有治
病的功能”,更不能证明所谓“孔子‘不治已病’”。张先生又拿来说一遍,那个“鲁国许止”
的笑话还没有停止,又来一个诸侯世子大叫“父皇”的笑话。张先生这样写来“鲁国的许止给父
皇许悼公进中药误致父亲死于非命一案……”。求求您了,张先生您还是好好请教一下你们学院
的古汉语或者历史方面的教师好不好。笔者已经告诉您了,这个“许止”不姓许,姓姜。这个
姜止不是鲁国的,是许国的世子,他的父亲悼公是许国的国君。再一个,那个时候诸侯世子称做
国君的父亲不称“父皇”,父皇要到秦嬴政称始皇帝之后,才有“父皇”一说。而且连“父王”
都称不上,春秋时期,诸侯国君除了周天子称王,就只有楚国称王,还没有到战国末,是个人都
可以称王的年代!张先生您是在写论文不是在写《大话西游》!

张先生引用《论语》中的三段文字,要说明孔子治“治未病”的方法。一一看来,哭笑不得。
哎,可怜的孔老夫子,死了两千多年了,到如今仍然地下不得安生!

张先生《治》文中用来说明所谓孔子“治未病”的第一个方法“讲究饮食卫生”的文字是这
样的:“《论语·乡党》记载,孔子很讲究饮食卫生。所谓‘食不厌精,脍不厌细。食饐而餲,
鱼馁而肉败,不食。色恶,不食。臭恶,不食。失饪,不食。不时,不食。割不正,不食。不得
其酱,不食。肉虽多,不使胜食气。惟酒无量,不及乱。沽酒市脯,不食。不撤姜食。不多食。’
吃饭要细嚼慢咽;不吃变色、变味、变质的鱼肉;食物不煮熟不吃;吃饭要守时间,不吃零食;
不酗酒;不把酒和干肉、果脯一起吃;每餐都必须吃姜;不暴饮暴食。孔子如此讲究饮食卫生,
其‘治未病’的作用至今也是无可挑剔的。”

还是先看看把张先生引用文献的理解是否正确,免得先生张先生再来个“你不知道不代表别
人不知道”的盛怒。

“食不厌精,脍不厌细。”张先生的理解:“吃饭要细嚼慢咽。”错了!意思很明显:粮
食尽量精,肉食尽量细;就是说食物越精细越好。

“食饐而餲,鱼馁而肉败,不食。色恶,不食。臭恶,不食。”张先生的理解:“不吃变
色、变味、变质的鱼肉。”三句并一句,马虎说得过去。不过,第一句已经讲了食、鱼、肉变质
的不吃,后面接着说色恶、臭(同“嗅”)恶,颜色不好看,气味不好闻,应当另有所指:不吃
颜色不好看的,不吃气味不好闻的。呵呵,孔子看来不会吃臭豆腐了,如果孔子生活的年代有臭
豆腐的话。

“失饪,不食。”张先生的理解:“食物不煮熟不吃”,又错了,应该是烹饪不得法或者
是做得不好的不吃。

“不时,不食。”张先生的理解:“吃饭要守时间,不吃零食。”吃饭守时,基本说得过
去。只是不知道那时候有没有什么类似话梅、巧克力、薯片的零食。这句话显然是说:不是吃饭
的时候不吃。跟吃零食没有什么关系。
\footnote{“不时,不食。”也有人理解为不是当季食物的不吃。考虑当时生产力低下,反季
节食物几无可能,本人倾向理解为不是吃饭的时候不吃。}

“割不正,不食。”张先生于此沉默。食物切的不合规矩不吃!

“不得其酱,不食。”张先生再次沉默。没有相应的酱不吃(那东西)!

“肉虽多,不使胜食气。”张先生还是沉默。肉虽然多,量不要超过吃的粮食!

“惟酒无量,不及乱。”张先生的理解:“不酗酒”,意思还差那么一点点:只有酒不限
量,不喝醉。

“沽酒市脯,不食。”张先生的理解:“不把酒和干肉、果脯一起吃。”您真的该去向
《论语》超女于丹女士讨教一下,虽然于女士把“小人和女子”理解成小孩子和女人,还不至于
把“沽酒市脯”解释成“酒和干肉、果脯”!沽酒,买来的酒;市脯,买来的肉脯。这句是说:
酒坊买来的酒和集市上买来的肉脯不吃!

“不撤姜食。不多食。”张先生断错了句读,一句变两句:“每餐都必须吃姜;不暴饮暴
食。”实际上是完整的一句“不撤姜食,不多食。”意思无需解释就已经很明白了。

张先生把《论语》这句话,信口胡言解释一通,下了一个结论:“孔子如此讲究饮食卫生,
其‘治未病’的作用至今也是无可挑剔的。”张先生您连文献都没有读懂,您就敢下这样的结论?
这就无可挑剔了?“食不厌精,脍不厌细”、“割不正,不食”有“治未病”的作用?这段文字
中心意思根本不是什么讲究饮食卫生,这是孔子要推行“礼教”而已,事事符合“礼”的规范,
只不过某些做法现在看来有点符合饮食卫生要求而已。

佩服死您了,您也真敢歪曲孔子的本意,硬生生凑您的主张!

张先生用来说明所谓孔子“治未病”的第二个方法“生活要简朴”的文字是这样的:“《论
语·泰伯》记载孔子的一段话说:‘菲饮食而致孝乎鬼神,恶衣服而致美乎黻冕,卑宫室而尽力
乎沟洫。’这意思是说,人类平常的饮食应该简单,祭祀则可以丰盛一些;平常穿衣服要简朴,
参与祭祀活动则可以穿得华丽一些;房子可以修得小巧简陋一些,但是沟渠一定要尽力修得宽阔。
众所周知,沟洫狭窄必致污泥浊水排泄不畅,污泥浊水排泄不畅必致蚊蝇滋生,流行病肆虐,因
此,孔子主张‘卑宫室而尽力乎沟洫’实在是有相当好的防疫意识的。”

笔者在以前文章评论张先生另一篇文章时,已经指出《论语·泰伯》这段话是孔子赞美大禹
的。既然张先生视而不见,那么笔者就勉乎其难做一回《论语》快男来教教张先生。这段话完
整原文是:“子曰:禹,吾无间然矣。菲饮食,而致孝乎鬼神;恶衣服,而致美乎黻冕;卑宫室,
而尽力乎沟洫。禹,吾无间然矣。”张先生孔子应该没有得罪于您吧!您是“外科医生”还是
“屠夫”,砍了夫子的头,去了夫子的势,快,您这刀真快!

“大禹啊,我真是没什么可说的了。他吃的饭食非常简单,但祭祀祖先和神明却十分丰盛。
他平时穿的衣服很破旧,但仪式上的礼服帽子却极为讲究。他住在低矮的宫室里,整天在外面尽
力修治沟渠水道。大禹啊,我们真的无法再形容他了”。孔子话显然是赞叹大禹虽然日常生活简
陋,对祭祀鬼神、举行仪式、关心民众疾苦一点也不马虎。到了您张先生的笔下意思就变成了
“人类平常的饮食应该简单,祭祀则可以丰盛一些;平常穿衣服要简朴,参与祭祀活动则可以穿
得华丽一些;房子可以修得小巧简陋一些,但是沟渠一定要尽力修得宽阔。”孔子赞叹大禹的话,
变成了孔子教训人类的说辞。况且,“卑宫室,而尽力乎沟洫”是指着大禹治水的传说而来的,
跟排放污泥浊水根本不搭界。维扬一带有方言说:“眼睛一眨,老母鸡变鸭。”张先生您这文字
戏法还不到家!

佩服死您了,您也真敢砍了孔子的头,去了孔子的势,硬生生凑您的主张!

张先生用来说明所谓孔子“治未病”的第三个方法“勤洗衣服,不要偏食,不要久坐在一个
地方”的文字如此:“《论语·乡党》记载孔子的言论说:‘齐必有明衣布。齐必变食。居必迁
坐。’讲的就是这些道理。”

“齐必有明衣布。齐必变食”这里的“齐”是“斋”。斋,繁体字写作“齋”,齐,繁体字
写作“齊”。斋、齐,形近相混。而且张先生引文少了一个句读,应该是:“斋必有明衣,布。
斋必变食。居必迁坐”。既然您引用这段文字,不要以为你知道了,别人也就都知道了。按照上
面的惯例,张先生您这里应该解释一下啊。既然这里没有按照上面的惯例解释一下,那我就可以
照您“但也算是情理之中”推论一下,您根本就没有看懂这段文字!斋,是古人临祭之前的一种
仪式,也就是后人所谓的斋戒。“斋必有明衣,布”,斋戒期间必须穿洁净光亮的衣服,布的
\footnote{有人说春秋时期棉花还没有传入,是没有棉布的。我国种麻织布时间非常早。麻布也是布。}。
古代贵族日常衣服采用丝绸、毛裘做的,这里意思就是斋戒期间不穿日常丝绸、毛裘的衣服,而
穿布衣。“斋必变食”,斋戒期间必须改变饮食,也就是不吃日常的食物,日常食物有些什么?
酒、肉、薤之类的东西,那就是说吃点蔬食。“居必迁坐”,(斋戒期间)静居必须移动坐的地
方。古人房屋有内外寝之别,内寝又称燕寝,是日常居处。斋戒必居外寝,外寝称正寝,斋与疾
皆居之。请教张先生您这段文字的三句话,哪个跟您所谓的“勤洗衣服,不要偏食,不要久坐在
一个地方”有联系?根本不用“也算是情理之中”的推论了,您根本就没有看懂这段文字的意思!

佩服死您了,您也真敢不懂装懂,硬生生凑您的主张!

张先生,您这篇文章如同《孔子不信医巫考》一样,浑身是洞,我就一万个不明白,您怎好
意思拿出来示诸广众?您真以为您是谁?您比不了陆象山,您也真敢“六经注我”!越发佩服得
紧。

您这样乱解经典,生拉硬扯,说你是“文科傻妞”真是对您的抬举,对“文科傻妞”的侮辱!
