\chapter{读张功耀先生《关于废除阴历的新设想》}
\blfootnote{* 2006年10月8日发表于《新语丝》,申明:本人对本文全部文字负责,与方舟子以及新语丝网站无关。}
\blfootnote{* 本文网址 http://xys.org/xys/ebooks/others/science/misc/yinli10.txt}

张功耀先生在《由“2033年问题”而论废除阴历》
\footnote{http://xys.org/xys/ebooks/others/science/misc/yinli.txt}
提到“为此,笔者根据‘废除阴历而不废除24节气’这一特点,已经对废除阴历
之后,如何重新安排我国民俗节庆日,做出了一个修订方案。是作已经发表于
《自然辩证法通讯》2006年第4期。”。而张先生这篇文章和《再说“2033年问
题”》\footnote{http://xys.org/xys/ebooks/others/science/misc/yinli5.txt}
一再出现对历法的错误理解,想来因为不是正式论文,严谨不够也是有可能的。
于是在网上搜索了想看看张先生在2006年第4期《自然辩证法通讯》的文章,
发现农历网有转载
\footnote{http://bbs.nongli.com/dispbbs.asp?boardID=2\&ID=4128\&page=1\\http://bbs.nongli.com/dispbbs.asp?boardID=2\&ID=4129\&page=1},
题目:《关于废除阴历的新设想》,并且明确注明来源于《自然辩证法通讯》
省略了最后一部分。“北大科学史与科学哲学”网站上截止本文成篇有18篇张先
生的文章\footnote{http://hps.phil.pku.edu.cn/search.php?q=张功耀\&t=author\&o=cat-date}。
其中就有这篇文章\footnote{http://hps.phil.pku.edu.cn/viewarticle.php?sid=2078},
没有标注来源,加入日期是2006年8月27日。虽然没有明确实张先生本人递稿还是转载,
考虑到“北大科学史与科学哲学”网站的学术性的可靠,我有理由认为该网站上刊登的张
先生这篇文章是原文,应该是学术性论文吧。仔细读来,真有些不知是啥味道。

首先看文章标题就有问题。用“阴历”来指称农历(按照现行官方的说法),在民间或者
非学术场合到也符合习惯,作为学术论文来说,显然不合适。阴历作为学术名词,与阳
历、阴阳合历一起是大类名词,划分的标准是按照据历法制定的依据:月球运动、太阳
运动还是兼顾二者的运动。在学术上阴历不能用来特指农历。我国的农历可称一种特殊
的阴阳合历:普通阴阳历性质和特殊的太阳历性质的集合。普通阴阳历只有闰月和固定
的闰周(属于平气)。农历虽然早期采用过平气,公元1644年发布的《时宪历》以官方
地位完全确定了定气制历(定气的应用远在此之前),置闰完全是通过太阳和月球运动
的实际天象来确定的。(实质上取消了闰周,所以,说现行农历“19年7闰”是不正确的。)
现行的农历中的二十四节气(包括七十二候在内)是真太阳历,是直接通过测定太阳位
置来确定的。

文章第一句:“汉武帝太初元年颁行《太初历》奠定了流传至今的阴历。”这句论述很是
成问题。现行农历的框架远远早于汉代,考古资料表明最迟商代殷墟时期就已经形成了
农历(阴阳合历)的几个根本历法规则。董作宾根据殷墟甲骨文中记载的商王祭祀先王
先公的日期整理发表的《殷历谱》显示商代晚期商王以旬(十天)为周期轮流祭祀先王
先公,历年长度为回归年长度,这是一个阳历体系。商代甲骨文刻辞、青铜器铭文多次
出现“十三月”表明存在一个阴历体系且是年末置闰。这表明殷墟时期的历法已经是阴阳
合历体系了。虽然也有人根据甲骨文研究认为商代晚期就已经开始采用年中置闰的规则,
但争议比较大。周代实行年中置闰是没有疑问的。可见,农历的历法规则形成远远早于
汉代。《太初历》虽然有很多改进,本质上并没有改变之前历法的结构和大的规则。
《太初历》在历法演化历史上之所以非常重要,因为这是有明确历史记载的第一次官方
的历法学术规范。“奠定”二字是谈不上的。

文章的第一个部分的小标题是“废除阴历是我国志士仁人的一贯理想”。想来意思是说废
除阴历是张先生要完成仁人志士未竟的理想。仁人志士的理想就一定要完成吗?仁人志
士的理想就用永远合适吗?比如废除汉字改行罗马化拼写文字的主张。当年,钱玄同、
鲁迅、瞿秋白、陈独秀、胡适等人,连毛泽东都主张过废除汉字,这些人都可谓仁人志
士吧。不过现在我们哪个还会再提废除汉字?张先生如果您敢提,那么晚辈对你的景仰
真是“滔滔江水,黄河泛滥”了。

张先生又要论证这是“一贯理想”,于是从沈括、孙中山谈到到国民政府等等要废除农历,
一直提到了建国后。后面罗列不少人名,说都是主张历法改革的,这倒没什么问题。但
是:历法改革 = 废除阴历?按理,张先生是搞科学哲学的,不会搞清二者之间的界限
的。那么拿这些人放在这里作为张先生“废除阴历”的论据,又该如何解释?特别提高戴
文赛、竺可桢两位大家,在参考文献中列出了两人的文章,戴的文章没有找到。竺的文
章《谈阳历和阴历的合理化》网上有转载
\footnote{http://bbs.hanminzu.com/dispbbs.asp?boardid=131\&id=98088}。
仔细看来真是有趣的紧,竺文中先论述了农历历法的基础知识,再回答了一个问题:我
们旧历既已过时,为什么不直截了当完全用新历即西洋现行的格里高里(Gregory)历
法呢?竺先生对此谈了四点,没有一点是说要废除农历的,反而是在说现行农历的合理、
有用。竺文的后一半却是要改革现行的公历历法的。我就照实好奇到底了,竺先生的是
张先生到底是看过了,还是没有看过。要是没看过,那么张先生就不要拉来做大旗了;
要是看过了,如果不是张先生理解能力有问题,我还真想不出有什么其他理由来。

接着张文中说:“陕西省科协甚至成立了以章潜五老先生为代表的历法改革专业委员会”。
上网查询,这个历法改革委员会不是陕西省科协的,而是陕西省老科学技术教育工作者
协会(简称陕西省老科协)的。这个历法改革专业委员会是陕西省老科协的集体会员,
建于西安电子科技大学,他们自己用“历改委西电”的简称。一字之差,不敢揣度张先生
是有意还是无意漏了这一个“老”字。(农历网转载的时候这个“老”字是用括号括起来的。
不知是不是转载者给添进去的。)

这一部分的最后一段,张文中说:“几乎所有中国传统习俗的改变都实现了社会进步。”
张先生的言下之意可否认为是:几乎所有中国传统习俗都阻碍了社会进步?这个我就不
多说了,说开来免不了又是一场笔仗。

文章第二个部分的标题:“为什么要废除‘阴历’?”看了这个标题,张先生是要论述理由
了。我在想,文章的第一部分算不算“废除阴历”的理由?

第二部分第一句,张先生这样说:“‘阴历’既不是‘月亮历’也不是‘农历’”。农历不是本
来就不是阴历。学术上通常都用“阴历”或者“太阴历”,张先生硬用“阴历”指农历,只好
用这个不常用的“月亮历”的名词。“不是‘农历’”的说法,想来在张先生的心目里面有一
类历法被为“农历”,请张先生给定义吧。不然我也不明白张先生的农历到底应该是什么
样子的。农历之所以被称为农历有历史原因。我国历史上“天命”观念和“以农为本”的重
农观念,决定了古代历法有两个作用,一是为“知天命”证明政权的正当性,二是求丰年
稳固政权的基础。古代历法中为“知天命”而包含的许多星占、气占的牵强附会的迷信东
西;为求丰年而包含了大量和农业生产有关的气象知识和经验(本质上,气象现象的起
因还是日地月系统运动)。封建社会结束之后,“知天命”必然被抛弃,求丰年还是保留
着。行公历之后,称原有的历法为农历是很自然的。

紧接着张说:“在世界现行诸多的历法体系中,它也许还是一种最不合理的阴阳合历。”
张先生这个观点是对农历的严重误解。从本质上说,如果不考虑技术手段限制的话,现
行的农历是唯一一部能够确保历法与日地月系统运动完全吻合的历法,这是因为现行农
历定年的年长、节气、朔望等参数全部是依据实际测量得到的,不存在历法本身系统性
误差。其他阴阳合历由于存在闰周,比如张先生提到的希伯莱历,无论如何精密,也存
在闰周带来的系统性误差。张先生提到了哈吉来历,也知道哈吉来历不是阴阳合历,那
我就不知道了,这跟“农历是不是合理的阴阳合历”有什么关系?

随后,张文写道:“与夏历最相近的是以色列人所使用的希伯莱历。”这里又冒出一个
“夏历”名词,从此之后一直到文章的倒数第二段都用“夏历”不用“阴历”。奇怪也哉,前
面不声明,后面不说明,突然换了一个名词。单不论是否能把现行农历称为阴历,这样
的行文方法少见,就像写一篇文章,前面李二长、李二短,后面突然李二不见了,满眼
都是二狗子如何如何。估计是复制粘贴大法的痕迹。

再下来,张先生说:“与希伯莱历相比,夏历置闰的“无中气规则”不但复杂,而且已经
面临着严重挑战。”意思两个,农历置闰规则复杂,二是这个规则面临严重挑战。先看
第一意思。为了对希伯来历有个基本了解上网找了找,找到了
The Department for Jewish Zionist Education of The Jewish Agency of Israel的
网站\footnote{http://www.jafi.org.il/education/FESTIVLS/calendar/index.html}。
虽然也很简略,发现张先生描述希伯来历的一个错误,张先生说:“每逢闰年,须在希
伯莱年历的第11月(Shevat)之后增加一个月(29天或30天)”,该网站在Adar(12月)
这个月下面用红色写了这样一句话:In a Leap Year, an extra month is added here,
and this thirteenth month is called Adar Sheni (the Second Adar). When that
happens, Purim is observed in Adar Sheni. 我理解释这样:逢闰年,多出来的一个
加在Adar月后面称为Adar II,闰年的话,Purim节在Adar II月里。我不知道张先生的
资料是哪里找来的,居然说闰月至于11月之后。虽然希伯来历置闰看来简单,但是历月
长度安排一点都不简单,有6种组合,这样年长也有6种,平年3种,分别为:353,354,
355日,闰年3种:383,384,385日\footnote{http://webexhibits.org/calendars/calendar-jewish.html}。
具体怎么排定希伯来历我还是没有找到资料,如果有谁给点资料不胜感谢。可以看到希
伯来历一点都不简单。再看第二层意思。张先生的所谓挑战是什么呢?2033年问题。
这是个“伪问题”,是张先生完全不了解农历历法规则造成的。这个不再谈了,参见新语
丝网站文章。

再接着,张先生说:“与伊斯兰宗教历法和希伯莱历法不指导世俗生活不同,中国的夏历
从一开始就是充分世俗化的。”穆斯林和犹太人的节日虽然大多数是宗教的,但是宗教节
日仅仅就是宗教的?跟世俗生活一点都没有关系?这个怎么也是说不通的吧。这两个历
法中的节日我知道不多,只知道穆斯林有古尔邦节和开斋节,犹太节日知道有逾越节,
都不大清楚具体情况。那么来看看现行公历吧。现行公历是不是一种宗教历法?我想不需
要我多说了。那么我们那么起劲过2月14日的情人节,12月25日的圣诞节是为宗教信仰?
显然宗教节日深入世俗生活当中之后,就不会跟世俗生活毫无关系。再说历法最开始并
不是为了世俗生活而生产的而是为了原始宗教的祭祀需要,祭祀在古人来说是头等大事,
祭错了神还了得?殷墟甲骨文卜辞最多的都是关于祭祀的,卜问祭祀一定都要问日期合
适不合适。中国历法的世俗化原因,是因为中国文化中人本主义思想起源甚早,“天听以
我民为听,天视以我民为视”。

再来,张先生说:“古代天文学家采用两种方法来确定年的长度。一种以恒星(如北斗星)
的视运动来确定;另一种是以太阳在黄道带上的视运动来确定。前者叫恒星年,后者叫
自然年。”中国古代的确实通过观测恒星间接得到恒星年(古人称为“周天”),但是不是
北斗星,而是观测夜半上中天(真天顶)的恒星。后者应该称为回归年(古人称为“周岁”,
回归年长度称为“岁实”),自然年一般是指一个历法年的自然长度,比如公历自然年为365天。
如果采用平气的话,恒星年跟历法编制没有多大关系,把回归年长按照时间平均分为24等分,
就是节气了。采用定气就不同了,节气点并不是都是可以直接测定的,需要通过观测太阳
视位置在天球背景的移动来推算,冬至点的移动必须通过岁差来确定。虽然古希腊很早就
发现岁差但是一直没有应用到历法制定上。反过来说明,现行定气的农历更加符合天文现
象。张先生又说:“24节气……它只反映地球围绕太阳公转的位置关系。把它解释为一种气候
描述,则明显是一种误导。”首先二十四节气名称很显然起源于黄河流域气候现象,这是
无法否认的。其次,二十四节气很早就已经称为纯粹的历法名词了,对中国很大一部分区
域的气候描述都是合适的。张先生说:“不用说在世界范围内用‘24节气’指导农事活动是刻
板的,在我国960万平方公里领土范围内用它指导农事活动也是刻板的。”我看刻板的是张
先生,一定非要从字面上来理解节气含义。再说二十四节气没有一个说,该插秧还是该收
割。民间各地农谚有许多,各地有各地特色,一点都不刻板。草根的民众大多时候都是比
殿堂的学者来的灵活聪明。

张先生笔锋一划:“夏历除了它误差大和不科学之外,在日常生活中,夏历所带来的社会
管理上的混乱也是显而易见的。”无论如何我也没有看出张先生怎么论证农历误差大,怎
么不科学的。太轻率了。农历编制的全部要素都是通过实际测量得到,无法实际测量的也
是通过积累的观测数据通过非常科学的插值法得到的。说闰月安排不科学那是张先生的误
解。张先生举了一个例子来说明:“夏历所带来的社会管理上的混乱也是显而易见的。”某
公闰五月出生,整寿那年没有闰五月,女婿不知道该在什么时候过生日,问张先生该怎么
办,张先生说无法回答。那我给张先生出个主意,很简单不想烦神就过公历的。非要想过
农历生日的话,去问问乡下老者,这世上不是只有张先生朋友的老泰山才会遇到这样问题,
也不会这几年才出现的问题。又印证了那句话:草根的民众大多时候都是比殿堂的学者来
的灵活聪明。

张先生说:“为了纠正这种混乱现象,我国公安部门在户籍管理中,试图推行以公历标准计
算生卒时间。”好像不是“试图”吧,这早就是法律规定了吧。就算老年人多以农历计算生日,
这也没有困难的,找本万年历来对一下不麻烦,不过千万不要贪图便宜买了盗版的或者是民
间的历法。紫金山天文台刚刚出了一本《一百六十年历表》起于1901,止于2060年,才45块
钱。话说回来,该查对农历生日早就查对过了,现在登记出生日期还会有几个父母会用孩
子农历生日去登记呢?还有什么混乱可言?

再来,张先生又说:“在人们的其它生活中,每逢夏历闰年,我们就有了13个月,可以过两
个端午节,或两个中秋节,或两个重阳节。令人不解的是,人们在责骂沈括的废历想法有些
‘怪异’的同时,却不愿纠正这种真正怪异的历法。”嘿嘿,过两个端午不觉得高兴而觉得怪
异的,估计还是象张先生这样的殿堂里的人吧。我倒是很乐意过两个节日。要是有闰正月
(可惜直到现在还没有过)的话,最高兴的是孩子,拿两次压岁钱。

张先生接来说:“其实这种怪异历法是具有社会危害性的。”说了一个例子:“比如,当两个
人约定:‘我们下个月11号见面’。结果可能出现两个人同时‘失约’。原因就在于,他们约定
的‘11号’既可能指夏历,也可能指阳历。”这种例子我都不想驳了,再驳下去,我都要成弱智了。

张先生又把一些迷信活动的帐算到农历头上了:“还有,民国初年规定的‘吉凶神宿一律删除’
的东西,近年来在我国的历法出版物中重新复活了,它给相信命理八字的人留下了广泛的活
动空间,从而造成了严重的迷信泛滥。”我真有点担心我要成为弱智了。出版物中出现“吉凶
神宿”那是出版社的责任,跟农历有什么关系。如果说农历给相信命理八字的人留下了广泛的
活动空间的话,天下的男人都要废除了才好,是男人都有强奸罪的犯罪工具。

张文的第三部分,论述了“废除夏历后,我们怎样过传统节日?”看了之后还是漏洞一大堆。
懒得一一驳斥了,拣几个主要的说说吧。这一部分的第四段中说:“我国当前这种以夏历为基
础来安排传统节日的习俗,实际上还没有在我国流行100年。”啥意思?张先生的意思是否说,
元宵、清明、端午、中秋、重阳等等节日,古人都不过,是从清末民初才有的?我是弱智,
我不明白。张文:“甚至一年的开头定在那一天也曾经存在相当大的差异。秦朝时的正月初一,
定在相当于现在夏历的十月初一。”年首月建的问题,有兴趣的自己查找资料吧。我累了。
虽然有过这样的情况出现,那跟现在的生活有什么关系?查公历看看,1582年10月4日之后
就是10月15日,缺了10天,跟现在的生活有什么关系?

张先生说:“至于有学者引述《榖梁传·桓公三年》中的一个说法:‘五谷皆熟,为有年也’,
来说明春节的起源,就更加不确实了。众所周知,‘五谷皆熟’的大致时间是在现今夏历的10月,
我国古代的先民,大概不会‘五谷皆熟’的10月就立即安排过年。如果非要在‘五谷皆熟’之后
立即过年,那也一定不能与我们现在的‘正月初一’相吻合。”请张先生告诉我,是哪个学者说
《榖梁传·桓公三年》这段里的年是过年的年?我骂他去。这里的“年”不是历法中年的意思,
是丰收的意思。甲骨文中“年”是手持刀割禾的图象,再有甲骨文多次卜问“受年”都是丰收的
意思。商代用祀用岁表示现在的年的意思。早期农业不发达,谷物一岁一熟,慢慢的年演变
成为现在的概念。不过周代建子,岁首在冬至所在月,相当于现在的农历十一月,张先生说
“‘五谷皆熟’的大致时间是在现今夏历的10月”这句不错,那么周人在收割完谷物之后就过年
却是很合理的。年从丰收意义变成现在年的意思,也正在这个时候了。虽然后来历法建寅,
岁首差了两个月,叫法沿袭下来很正常。再一个张先生说:“周朝是以六月为正月的”,张先
生,这是谁告诉你的?我砍他去。

就快结束了,张先生终于提出了自己的传统节庆日修订方案。不多说了,新语丝上异调先生
已经评论过了。

张先生洋洋洒洒,篇幅不短,漏洞不少。看来张先生不看书的毛病不是最近才养成的,至少
有了5个月了(从2006年4月算起)。
