\chapter{难道真的是我误读了张功耀先生?}
\blfootnote{* 2007年6月27日发表于《新语丝》。}
\blfootnote{* 申明:本人对【】之外的全部文字负责,与方舟子以及新语丝网站无关。}
\blfootnote{* 本文网址 http://xys.org/xys/ebooks/others/science/dajia8/zhongyi1057.txt}

zeroyear先生说笔者“关于历史时间认定上似乎是误读了张功耀的
文章”\footnote{http://xys.org/xys/ebooks/others/science/dajia8/zhongyi1046.txt},
我这也人有些较真的脾气,那来看看张先生关于历史时间的认定,
到底是不是我误读了张先生的文章?

张先生原文如下:(【】中文字为张先生《中国人民努力摆脱中
医困扰的一些历史线索》
\footnote{http://xys.org/xys/ebooks/others/science/dajia8/zhongyi1035.txt}原文引用)

【\fangsong 针灸学是在两个阶级、两条道路、两条路线斗争中发展起来的。
……到了清代后期,出于维护封建王朝的旧礼教,竟下令禁止针灸。

注解:清朝后期曾经禁止过针灸,这个历史估计许多读者都不清楚。作者说,
清朝废除针灸与旧礼教有关,这不免有些牵强。其实,针灸与旧礼教没有关系。
作者的这个提法明显受了“批林批孔”和“反击右倾翻案风”的影响。作者实际
要写的情况大概是指17世纪末的一次废医风波。

1569年,科学医学从澳门进入中国以后,中国人民终于看到了医学的曙光,
同时也看到了中医“平和藏拙”的本性。1693年,康熙皇帝得了疟疾。尽管中医
关于疟疾的“偏方”、“秘方”、“神效方”汗牛充栋,但是,所有中国的宫廷
御医和民间名医都对皇帝的疟疾没有任何有效的作为,而葡萄牙医生刘应(Mgr
Claudus De Visdelou)用不到两钱的金鸡纳霜,只两天时间就让康熙皇帝康复
了。于是,中医的无能和西医的神奇,在当时的宫廷里边便诱发了一“废除中
医中药”的风波。康熙皇帝也因此成了国家领导人当中的第一个“废医派”领袖
人物。\normalfont 】

连续的三段文字,第一段引用1977年出版的《针灸推拿学》原文,最后一句
历史时间明确是“清朝后期”。

第二、三段张先生的文字。第二段开头历史时间张先生也明确写道是“清朝
后期”,该段的最后一句“作者实际要写的情况大概是指17世纪末的一次废医风
波。”张先生难道不是把“17世纪末的一次废医风波”认作是“清朝后期曾经禁
止过针灸”吗?17世纪末至少应该在1660之后,1700年之前,这个时间段难道是
清朝后期?

第三段中,张先生写道“1693年,康熙皇帝得了疟疾。……诱发了一场‘废
除中医中药’的风波”。回过头来,看看第二段所谓的“17世纪末的一次废医风
波”指的不就是这个所谓“康熙得了疟疾。……诱发了一场‘废除中医中药’的
风波”吗?把二、三两段文字连起来看,张先生难道不是把康熙算在了“清朝后
期”?

难道真的是我误读了张功耀先生?
