\chapter{踏雪访梅}
\blfootnote{高中时代旧文,青涩年华记忆。}

赏梅的去处,要数着江南的几处胜迹了,金陵吴王坟,姑苏香雪海便是其佼者。
也曾诣吴王坟,也曾游香雪海;梅英如海,人影如潮。心里总觉着不若许久以
前踏雪访梅的意趣。

家乡的老辈人总以为从腊月二十三祭灶神开始算过年,除夕祭祖祝福,初一祭
天地,初五奉财神,直到元宵才算过完年。元宵节却不是正月十五那么简单,
正月十三就要扎青门,家家户户张灯结彩,十五闹元宵“一夜鱼龙舞”,十八晚
上收灯,十九开市,年才真的叫过完了。前前后后有一个月,我是受不了这样
的喜庆的。每年回家过年,也没有什么新意,只是父母年纪大了,回家看看,
聊表孝之道罢。初一到初五走亲访友,落不得半日闲功夫。那一年天公作美,
初一到初五都是晴天,走亲访友倒没有吃什么苦头。

初八阴了天,家乡人的说法是龙翻身。初十飘起了大雪,纷纷扬扬,真可称得
上一个“紧”字了。自小在江南长大,第一次才知道什么叫做鹅毛大雪。这是一
场江南少见的大雪,据老辈人讲百年也是难遇的。地上铺满了雪,柏树压折了
树冠,垂头丧气地立着。不过对我来说,这的确是一场好雪,不必做那些年后
的回拜,倒也清闲自在。捧着缭绕烟雾的清茶,诵诵诗书也不必红袖添香,自
有案头凌波侧目顾盼。或者隔着窗户赏看雪景,古人有打油诗云:“江山一笼
统,井上黑窿洞;黑狗身上白,白狗身上肿。”真不为过也。这样的清闲不知
过了几日。

一夜,坐于案边诵“疏影横斜”之句。忽觉耳边几声啁啾的鸣叫,甚是诧异。推
开窗户,清寒扑面而来,沁入心脾使人神清气朗。雪停了,天晴了,银轮悬在
苍穹,乱琼散于原野,交相辉映,人间的灯火黯然失色,不忍多看。今夕何夕,
天上人间。广寒宫中的嫦娥是否起舞。熄了屋里的灯,静静地享受这荡漾清辉
的世界,清寒往肌肤深处行去,却不甚冷。“疏影横斜水清浅,暗香浮动月黄
昏。”不觉诵出了声,惊醒了自己的梦。想起梅花一定开着。便着了木屐扎上
草绳,出了门,向屋后不远处的小山走去。

路上的积雪蓬松而柔软。足有一尺多厚。使劲跺跺脚,发出了“嘭、嘭”的声
音,惊了栖于枝头疑是白日的麻雀,扑喇喇地乱飞。心底遗忘许久的童年的
顽皮露出了得意的笑容。只是双脚却深深陷在草地里。皎洁的月色下,树木
的身影怪异地映在雪原上,有如妙手偶得的水墨。

这山其实称不得山,约摸十多层楼房的样子,只能算是个大土堆。山脚有一
条婉延曲折的小溪,夏日里农家的孩子常在那儿捕捉小鱼小虾。溪头的垂柳
遮出一片浓荫,远处不时传来牧童的笛声,那是个读书的好去处。此刻溪头
一样积了厚厚的雪,只有溪水仍然默默地流动,显示出一种墨玉的颜色,看
起来有些静秘。小山有个鲜为人知的名字“皀嶺”。那是一次夏季暴雨冲出一
块怪石,上面刻着这两个字,字迹早已漫漶了,依稀能辨认出来,遒劲而雄
浑,颇有史晨之遗风。边款荡然无存,不可考当年的情致了。也许当初本无
边款,倒是我这个后来人凭空多了几分遗憾而已。翻翻《说文解字》有“皀,
穀之馨香也。”《唐韵》“皮及切,音近弼。”而《集韵》则曰“虚良切,
音香。穀香也。”这“皀嶺”之名,念“弼还是念“香”已无从考据。想来当初指着四
周的稻田而言,如今恰成谶语,或许始建之时就已了然免不了的结局了。溪
上有座汉白玉的小石桥,已经残破不堪了,而此时那平日里掩饰不住的往昔
精致和周围田野中偶尔能看到的柱础,全部的怀旧都覆盖在积雪之下,兴与
衰的感叹了无痕迹。

小心翼翼地走上山去,不能闲庭信步了。得看着脚下,辨清道路,跌了眼镜,
是无处可寻的,颇有些狼狈。其实,路在平日里就是没有的。大雪覆盖之时,
更是难寻。鬯岭之上另是一番情趣。几株梅树错落地立于乱石之中,然而此
时的乱石失去了棱角,涌动出奇怪的模样。盘曲的枝干积了雪,添了几分婉
转与圆润,蓦然横空,映在银色的雪原之上有如矫龙游于云海之中。淡淡的
清香随着清寒的空气飘了开去。花是不容易辨认的,洁白而柔软的花朵与冰
雪凝成一片,真不知道是梅花香还是那冰雪更芬芳。有如此的冰雪,有如此
的清馨,真当起舞了。

邀明月,邀寒梅,与我同舞。林逋,你若有灵,来这荒郊雪原与我同访这孤
傲的梅花吧!寒梅,你若有知,当为我开放!今夜,有月如水;今夜,有雪
如琼;今夜,有梅如影。他年还有这样的佳景否?

此后多年,也曾到过许多地方,喧闹的梅海,喧闹的人潮,于我总觉得不关
什么事体,有谁同我去顾盼那鬯岭上的孤梅。每年回家乡总要访那几株梅树,
只是何处能得瑞雪,何处能得满月。
