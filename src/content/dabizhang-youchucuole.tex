\chapter{又出错了}
\blfootnote{* 2006年10月6日发表于《新语丝》。}
\blfootnote{* 申明:本人对本文全部文字完全负责,与方舟子以及新语丝网站无关。}
\blfootnote{* 本文网址 http://xys.org/xys/ebooks/others/science/misc/yinli8.txt}

\subtitle{——评张功耀《再说“2033年问题”》\footnote{张功耀原文网址 http://xys.org/xys/ebooks/others/science/misc/yinli5.txt}}

在本人以及hedian、异调的文章发布之后,张先生又发布了《再说“2033年问题”》,
可惜张先生仍然没有弄清楚几个问题。张先生提出“废止中医”、“废止阴历”主张的
行动,我没有什么意见。有看法就说出来,都是可以讨论的。我所针对的是张先生
在历法认识上的一些错误,或者说是方法态度吧。我还是那句话:“在自己熟悉的
领域踏踏实实做点事情是应当的,在自己不熟悉的领域小小心心看点东西是要紧的。”
这里我不打算讨论是否应该废止阴历,只是要指出张先生《再说“2033年问题”》的
错误。张先生不仅没有吸取看错书的教训,这回连书也不看(至少看了一本盗版书),
就开口说话。

张先生说读到了一本盗版历法书,张先生应该交代一下,这本盗版是盗印的盗版还
是彻头彻尾的盗版,盗的是谁的版。如果盗的是紫金山的版,闰十一月是自然。或
许就是民间历法盗书号的盗版也不是说所有民间历法都编错了,民间到底还是有高
手的。

张先生说:“‘回归年’概念在中国古代不是十分确定,尤其是在西汉时期观察到分
至点岁差现象以后,变得更加混乱。”这句话完全是臆测。中国古代虽然没有“回
归年”的名词,但是有个“岁实”的名词,概念十分明确,是太阳视运动在天球上连
续两次通过冬至点的时间间隔。而回归年现代定义为:太阳视运动在天球上连续
两次通过春分点的时间间隔。除了天文参考点不一样,这个“岁实”跟“回归年”没
有本质区别。只不过在古代由于技术限制导致精度不高。中国传统历法编制要做
的第一件事情就是定岁长,早期采用平气,根据岁实的平均值来确定年长。在发
现岁差之后,不久历法就改平气为定气,以实测“岁实”为年长。这里有个矛盾,
历法要准确就必须采用实测岁实,然而未来的“岁实”无法测定,古人采用办法和
现代人的方法原理一致,非常科学,根据积累的数据,以插值的方法来确定未来
年份的岁实。这也就是为什么中国古代特别重视每年冬至和夏至的圭表测影(通
过测量太阳高度来确定冬至和夏至的准确时刻)。

张先生说:“‘冬至规则’……笔者的确没有注意到,什么时候引入了这一条规则。
它大概是遇到‘2033年问题’以后,有些历法研究者新引进来的。”这就是笑话了,
冬至是十一月的中气,如果冬至所在月不是十一月,那还是中国的历法吗?张先
生武断得可爱。张先生又说:“2033年将7月置闰,只差一天就可以满足‘冬至所在
的月为11月’这个规定,如果不是为了这一天,引入一条360年一遇的‘冬至规则’,
就是多余的。可见,异调先生引述的以‘冬至规则’优先,辅以‘无中气规则’作为
置闰的基本规则,是临时性的。这些规则并不能证明阴历历法的科学性。”那我要
告诉张先生的是,“冬至必定在十一月”实际上是历法排定的一个必然结果,根本
不是什么规则,下面细说。

张先生仍然没有搞清楚农历历法排定的方法。中国传统历法排定次序,异调先生
已经说得比较清楚,不过还有一些疏漏。这里,我不惮其烦再说一遍,希望张先
生仔细看看。首先确定两个冬至之间岁实长度,然后排定历月,再看两个冬至之
间除去两个冬至所在月外有多少个整月(必定是十一个或者十二个。虽然民间以
正月初一为岁首,但是排定历法却是以两个连续的冬至为始终。)如果是十一个
整月,则无需置闰,即使出现无中气的月也无需置闰。如果有十二个整月则需置
闰,此时置闰依照“无中气规则”,而且闰前不闰后。月序的确定,是以冬至所在
月为子,依照地支顺序,以大寒所在月为丑,以雨水所在月为寅……(更早是按照
北斗星杓柄所指向的方向来确定,实际上是一回事。)排列下去,到下一个冬至
所在月又回到了子。由于汉以后的历法都是建寅,以寅月(就是雨水所在的月)
为正月,这样冬至所在月必然是十一月。可见所谓的“冬至规则”并不是什么规则,
而是历法排定方法自然出现的结果。

张先生接着说:“‘无中气’的2033年7月和2034年正月可以不置闰,有‘节’有‘气’
的2033年11月反而要置闰。这本身就是阴历历法的一个反常问题。”我不知道张
先生这个2033年11月是指冬至所在月,还是冬至后所在的月。如果是冬至所在的
月,自然有节有气不需要置闰。如果是冬至后的一个月,那是第十二个月,不是
11月,想来张先生应该指的是第十二个月。我更不知道,张先生又是看到的哪一
本盗版历法。明明异调先生已经指出了,2033年冬至之后的那一个月没有中气。
张先生就是视而不见。非要坚持所谓“有‘节’有‘气’的2033年11月反而要置闰”,
这倒是确确实实很反常。
