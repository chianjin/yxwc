\chapter{彻头彻尾的伪“2033年问题”}
\blfootnote{* 2006年10月4日发表于《新语丝》,仅做排版调整。}
\blfootnote{* 本文网址 http://xys.org/xys/ebooks/others/science/misc/yinli2.txt}

\subtitle{
    ——评张功耀《由“2033年问题”而论废除阴历》\footnote{张功耀原文网址 http://xys.org/xys/ebooks/others/science/misc/yinli.txt}
}

【括号内宋体为笔者的文字,括号外仿宋体为原作文字。申明:本人
对括号内全部文字完全负责,与方舟子以及新语丝网站无关。】

【看完张先生的大文章,嘿嘿,题目就有问题,我国的农历属于阴阳
合历,民间称为阴历尚可,堂堂张先生张口“阴历”闭口“阴历”,就这
一点文章的严谨性就可疑。考古学表明最晚商代后期,我国就已经开
始使用阴阳合历。有几个重要的证据:其一,董彦堂(作宾)根据的
甲骨卜辞的祭祀日期排定的《殷历谱》清楚表明殷墟时期,商代帝王
在安排祭祀日期上采用了以旬为周期的阳历历年;其二、甲骨文中随
处可见的“初吉”,“既生霸”,“既望”,“既死霸”的月相名词,特别多
次出现的“十三月”的甲骨刻辞,显然是以月相定历法月。其三、周代
青铜铭文多次出现“闰某月”表明,年中置闰的规则在周代已经确立。】

\fangsong 我国从汉武帝太初元年开始采用的阴历历法,繁复而且混
乱。早在11世纪,沈括就主张用“12气历”来替代阴历。由于沈括的
“12气历”不能确保与天文历法吻合,这个历法最终没有能够建立起来。
\normalfont
【“天文历法”是什么历法?人有历法,天何来的历法?这里所谓天文
历法,显然应该改为“天文现象”。不知是张先生的笔误,还是有意的
误导?我看这个问题不在于是否能够与天文现象吻合,沈括虽然涉猎
很广,但是沈括属于广而不深的杂家,虽然能够提出十二气历的框架,
但是并不能解决历日计算具体问题即“历理”。不解决这个问题如何推
定历日安排?任何历法要与天文现象完全吻合,根本就是不可能的,
无论如何都有误差。只不过哪种历法更符合天文现象而已。】
\fangsong
于是,我国只好在太初历的基础上修修补补,凑凑合合,平均不到20
年修改一次,一直用到了今天。
\normalfont
【中国的历法并不是张先生所说,修修补补,期间最重要的三次变革,
一次是周代的年中置闰,二是晋代发现岁差,三是南北朝时期改“平
气”为“定气”(即平太阳时和真太阳时)。随着观测水平的提高,历
法的修改是必然的。现行公历也不是一开始到现在就没有修改过。
况且天体运动方式在大尺度上仍然在缓慢变化。人类无论如何也造
不出一部万年不改的历法。】

\fangsong
本来,1912年和1931年,我国科学家曾经两度提出过废除阴历之议,
最终都没有成功。这些不成功,不在于科学上不成熟,而在于民俗
的存废争论。
\normalfont
【废历并不是行政命令能够一纸公文解决的。国家废农历只能废除
国家学术机构的农历颁布。我国自古就有官方历法、民间历法的传
统。废除了官方农历,民间农历怎么废除?废除不了民间历法,农
历历法更混乱。指望废农历一了百了,不过自欺欺人。】
\fangsong
其实,指导世俗生活是不需要两种历法的。世界上没有任何民族在
指导世俗生活方面使用过两种历法。这与信仰伊斯兰教和信仰犹太
教的信众,指导宗教生活分别使用哈吉来历或希伯莱历,指导世俗
生活则一律使用阳历,是完全不同的。
\normalfont
【我没有听说韩国明确废除农历,是不是我孤陋寡闻?请张先生赐教。】

\fangsong
我国从1912年开始实行的“二历并存”,所导致的社会混乱是有目共睹的。
\normalfont
【我不知道这“所导致的社会混乱”是什么?请举出具体的例子。不
是这么一句话就可以混过关的。】
\fangsong
就科学层面说,由于我国没有废除阴历,使得我国现代的科学家在
使用阴历中,面临着一个迫在眉睫而又无法回避的“2033年问题”。
按照现有的阴历置闰方法,阳历2033年(阴历壬子年)存在两个
“有节无气”的月,7月和12月。根据置闰的“无中气规则”,
\normalfont
【“就科学层面说”这个高帽子并没有张先生说的一个硬伤。张先生
首先请搞清楚一点:这个“无中气规则”并不是唯一的置闰规则,如
果仅仅认为“无中气”规则就是唯一的规则未免小看古人了。】
\fangsong
2033年应该既闰7月,又闰12月。我国已经出版的历法书,有意回
避了这个问题。把闰月置在“闰7月”。
\normalfont
【再请张先生搞清楚另一点,先生看到的历法是国家学术机构公布
的历法,还是民间版本的历法。很简单上网搜索一下,的确有民间
历法置闰第八个月,即是张先生所说的闰七月。再仔细看看,紫金
山天文台的《大众历法》和台湾中央研究院计算中心,都把这一年
置闰于第十二个月,即闰十一月。却不是张先生看到的所谓闰七月
更不是闰十二月?莫非两岸官方历书都错了?如果张先生坚持两岸
官方历法都是错的,闰七月是正确的,那算我没说。】
\fangsong
可是,这样置闰,不但没有解决问题,而且连带引出了五个新的问题:
\normalfont【于是乎,张先生根据他看到的民间历法,提出了这五
个问题。既然两岸官方历法都不是闰七月,何来张先生的所谓五个
问题?张先生是真的没有看到两岸官方的历法还是有意回避两岸的
官方历法?本人不想揣测。】\fangsong

一、壬子年12月出现了三个节气(大寒、立春、雨水)。通常一个
阴历月只能安排一个节气或两个节气,这里出现了三个节气。

二、雨水节出现在12月末,于自然历严重不合。通常情况下,隔年
立春,造成来年“寡年”就比较少见,2033年居然把来年的雨水节都
提到了“旧年”的12月了。这样荒唐的历法,在我国历法史上大概是
绝无仅有的。

三、由于7月和12月本应该分别置闰,而现在却只对7月置闰,结果
导致来年(2034年,阴历癸丑年)应该补充“闰正月”才能与节气安
排相合。可是,历法学家没有安排2034年“闰正月”。

四、由于2034年本应该“闰正月”,却没有在历法中安排,于是,
2034年的阴历整个全年都是“气”在月前,“节”在月后。这种局面
一直延续到2035年的阴历5月才恢复正常。这意味着,所谓“2033
年问题”实际影响所及从2033年阴历7月一直到延续到了2035年阴
历5月,计22个阴历月。

五、如果保持2034年为正常,就必须在2033年一年闰两个月
(7月和12月),这就使我们陷入双重荒唐。第一个荒唐是,一个
阴历年过两个闰月,使一个阴历年的长度由360天增加到420天;
第二个荒唐是一年之内过两个12月。

我国古代天文学家制定阴历历法出现错讹,限于当时的科学技术
水平。自从有了刻卜勒模型以后,天文历法的建立已经有了坚实
的天文学基础。发展到现在,年长的测量已经精确到了纳秒的水
平。而我国却因为固守一个不科学的旧历法,在世界人民面前闹
笑话,冒出了古往今来都可以算独一无二的“2033年问题”。
\normalfont
【闹笑话的是张先生,拿一个根本就不存在的所谓“2033年问题”
的伪问题来说事。】
\fangsong
深入的计算
\normalfont
【谁的计算?张先生的计算?民间的计算?学术机构的计算?】
\fangsong
还表明,如果我们继续保守目前的阴历不变,将在2263年和3358
年继续出现类似情况。这就表明,如果我们这一代科学家要对后
人有个交代的话,就必须严肃认真地对待这个问题。【张先生自
己弄不明白农历历法编制的历理,何苦来摆出一副忧国忧民的脸
色来。】
\fangsong
现在看起来,解决这个“迫在眉睫”
\normalfont
【这火还没有烧到我们小民的眉毛,不过我先来放火烧一烧张先
生的眉毛。】
\fangsong
的问题的唯一办法,就是废除阴历,使阳历历法成为我国指导世
俗生活唯一的历法。
\normalfont
【现行的公历是否完全合理?平年2月28天,闰年2月29天,
7月、8月连续31天。合理?其实早有人曾经提出对现行公历的改
革,一年四个季度,每季度91天,每季度第一个月31天,其余两
个月30日,正好13个星期,这样12个月一共364天,平年年尾一
天不记月日,作为公共假日,闰年6月30日和7月1日之间增加一天,
一样不计月日,作为公共假日,每年1月1日定于冬至日。这样历
法岂不是更加完善?可巧,可巧,这里面竟然出现了一个所谓不
吉利数字“13”!哦,“魔鬼的历法”!】
\fangsong
否则,我们这一代科学家
\normalfont
【我不知道这“我们这一代科学家”是否包括张先生?反正不会包
括我】
\fangsong
将无颜以对世界人民和我们的子孙后代。
\normalfont
【草率废除农历,那才是上无脸面对祖先,下无颜面对子孙。建
国以后草率废除的传统还不够多吗?】

\fangsong
以往我国科学家的废历之议之所以未能成功,悉来自国内民俗学
者对移风易俗的担心。其实,民俗是历史形成的,也是可以改变
的。1914年,袁世凯强行把“过年”改成“过春节”,虽然形成了新
的民俗,却并没有给普通老百姓带来任何幸福。相反,我们废除
了男人蓄长发,女人包裹脚的习俗之后,天并没有塌下来。历史
事实表明,我国历史上的每一次移风易俗,都或多或少地引起了
社会进步。

但是,我们并不希望废除阴历给我国现有的民俗造成太大的冲击。
为此,笔者根据“废除阴历而不废除24节气”这一特点,已经对废
除阴历之后,如何重新安排我国民俗节庆日,做出了一个修订方
案。是作已经发表于《自然辩证法通讯》2006年第4期。我的修
订结果如下(下页)。
\begin{table}
\fangsong
\centerline{废除阴历后我国主要传统节庆日的修订方案}
\vspace{0.5em}
\zihao{6}\begin{tabular}{llll}
\hline
旧节日名称 & 新节日名称 & 旧节庆日期(阴历) & 修订后的节庆日期(阳历) \\
\hline
正月初一(春节) & 春节 & 正月初一 & 大寒后的第一个朔日 \\
元宵节 & 元宵节 & 正月十五日 & 立春后的第一个望日 \\
清明节 & 清明节 & 清明节 & 清明节 \\
端午节 & 端午节 & 五月初五 & 五月五日 \\
七夕节 & 情人节 & 七月初七 & 八月的第二个周日 \\
中元节 & 中元节 & 七月十五日 & 立秋后的第一个望日 \\
中秋节 & 中秋节 & 八月十五日 & 白露后的第一个望日 \\
重阳节 & 重阳节 & 九月九日 & 九月九日\\
\hline
\end{tabular}
\end{table}
\normalsize

这个方案是在全面废除夏历而不废除24节气,并对1900-2050年夏历
与公历的对应关系进行统计分析的基础上制定的。其中,春节、
元宵节、清明节、中元节和中秋节的设定,与目前流行的过节时间
安排100%地吻合。重阳节,旧时的设定本无其它含义,纯粹是以
数字定出的。为保留这个“九为阳极之数”的易学传统,也便于记忆,
笔者把原来的“夏历九月九日”改成了“阳历九月九日”。这个更改使
重阳节具有了“三阳开泰”(“阳历”、“阳极之月”和“阳极之日”)的
蕴意。估计这个更改很容易形成新的习惯。端午节,据梁朝吴均在
《续齐谐记》中所做的解释,是战国时的楚人为纪念屈原大夫投汨
罗江自杀而设的。但,屈原生活时的五月只相当于现在的阳历三月。
既如此,我们也就没有必要去追求它与夏历五月初五是否吻合的那
种安排了。把它安排在阳历的五月五日,不仅满足了“端午”、“重五”
的数字谐音,而且与屈原投江的实际日期更加接近。“七夕节”取自
古时人们观察到的一种星相,含有情人之间相互怀念之意。是故,
笔者将节日名称改成了“情人节”。由于星相是变化的,古时出现过
的星相,我们今天未必都能够看得到,抑或看到了也未必就是古时
的模样,这就是说,“牛郎织女相会”的星相,未必一定发生在“七
月初七”。因此,这个时间是完全可以重新确立的,而且不必局限
于星相。事实上,“牛郎织女相会”的星相不一定在每年夏历的“七
月初七”出现。根据笔者对1901年到1991年90年间夏历七月初七与
公历日期相对应的统计分析,把我国的“情人节”定在阳历8月的第
二个周日是最恰当的。

以往我国主张废除阴历的学者,在废历后大量更改了我们中国人
的习俗,从而引起了一些社会学家和民俗学家的异议。笔者的这
个废除旧历方案,基本不改变我们的习俗,希望它能够被全世界
广大华人群众所接受。

\normalfont
【所谓“2033年问题”完全是一个彻头彻尾的伪问题。农历是否
该废,这是完全是可以探讨的。但是张先生不看学术机关的历法,
而以民间历法的所谓混乱来混淆视听,要用这个伪问题来废农历
看来只不过想兜售自己的所谓历法改革方案。劝张先生一句,在
自己熟悉的领域踏踏实实做点工作是应当的,在自己不熟悉的领
域小小心心看点东西是要紧的。】
