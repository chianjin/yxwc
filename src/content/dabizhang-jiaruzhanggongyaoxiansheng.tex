\chapter{假如张功耀先生……}
\blfootnote{* 2007年7月16日发表于《新语丝》。}
\blfootnote{* 申明:本人对本文全部文字负责,与方舟子以及新语丝网站无关。}
\blfootnote{* 本文网址 http://xys.org/xys/ebooks/others/science/dajia8/zhongyi1105.txt}

假如张功耀先生扎实一点,仔细研读经典,搞清楚历史文献的真实背景和含义;

假如张功耀先生认真一点,不把以所谓“情理之中”推断当作事实;

假如张功耀先生大方一点,坦然接受批评;

那么对于张功耀先生的文章,我还能说什么?我又能写什么?

\mbox{}

我早就说过:“中医是否应该告别,可以讨论。”但是象张功耀先生这样生拉硬扯
算什么?乱解经典又算什么?别人批评到底当没当作一会事情?去年“历法”讨
论当中,张功耀先生除了想当然地为自己的观点辩护以外,接受过一点别人的意
见吗?

《孔子不信巫医考》的文章,我也撰文指出过其中的错误,张功耀没有看到?还
是不屑一顾?同样的错误再次出现,而且还多了一个错误。护中医的拉历史拉文
献,难道废中医的也就必须拉历史拉文献?在新语丝上,历史文献有功底的人应
该不少,怎么不见别人拉上作古的人?就算拉也要拉对人啊!

说实话,对于张功耀先生依据“科学哲学”来讨论中医的文章,很有理论水平。
即使不能说滴水不漏,也是见到多年学术功夫的。这不是很好吗?干什么非要在
自己并不擅长的方面搅来搅去?

也许,张功耀先生果真有点“历史癖”,把个追溯到黄帝这个子虚乌有人物的家
谱堂而皇之的挂在自己的博客上,下面八个大字:“报本祖宗,鞠躬尽瘁”,不
知所云。揣摩起来大概的意思是:报告本人的祖宗,我将鞠躬尽瘁”。但是“本
人的祖宗”可以简称为“本祖宗”的话,那么是不是还可以说:本父、本母、本兄、
本姊?进一步“本官”啥意思?是:“本人的官长”?这有很意思吗?张功耀先
生还兴致勃勃地讨论炎帝、黄帝该不该一块祭祀(《炎黄之祭源流述略》),
这篇妙文,同样拥挤着一堆对历史的误读和误解!这很有意思吗?

对于张功耀先生在中医方面的观点,我从来没有评论过。对于中医、现代医学,
自知没有水平参与讨论。我参与新语丝的讨论是从张功耀先生提出“废阴历”开
始的,而且我也只是对张功耀先生的在历史、文献方面的错误提出看法。

顺带说一句“文科傻妞”,如果这是侮辱性的语言,那么新语丝上面发过文章的
人,有几个可能又有吃官司的希望了!
